\chapter{Literature Review\label{chap:literature_review}}

\section{Technologies for Unaided Indoor Positioning}
\todo{Keep section or remove?}
Indoor positioning has been a topic of active research for the past decade with
the first research system using RF signals for distance estimation and indoor
positioning being produced by Microsoft Research (RADAR, 2000)\cite{RADAR}. Of
course, systems like Active Badge\cite{ActiveBadge} have been developed as far
back as 1992 to locate targets indoors but they have shown limited utility on
account of the requirement to deploy specialized sensors for detecting the radio
tags deployed within the system.

Since RADAR, a number of improvements have been made in the core algorithms and
technologies for positioning. Over time, indoor positioning has progressed from
simply using RF signals to using ubiquitous 802.11 Access Points as radio
sources. With the development of UWB (Ultra Wide Band) technology, very fine
grained ranging and tracking results have been obtained over target distances of
the order of a few hundred feet from the radio source.\cite{UWBRanging} 

However, the primary scale of interest for commercial exploitation of indoor
positioning and tracking is of the order of the size of warehouses and malls
(approximately of the order of $2500 m^2$) with an accuracy that is preferably
of the order of a few meters. Ultimately, the following goal needs to be met:

\begin{quote}
Create a low complexity, low cost indoor positioning system that is
maintainable, reasonably accurate and highly scalable.
\end{quote}

UWB, unfortunately, does not scale easily to such large spaces without requiring
that a significant number of sensors and repeaters be installed in the target
positioning area specifically for this purpose. It is also an expensive
technology. On the other hand, IEEE 802.11 AP based positioning has low
complexity and cost and it has a large installed base, thus making it attractive
for positioning purposes as a source of radio signals. 

Thus it is interesting to explore the limitations of accuracy, scalability and
maintainability of indoor positioning systems that are based on IEEE 802.11
Access Points.

Other technologies for indoor positioning include laser range finding, 
ultrasound ranging etc. %\todo{cite papers?}

\section{Indoor Tracking}
%\todo{Keep section or remove?}

Once having worked with indoor positioning systems, it is but natural to ask:
Can we extend all the technology developed for indoor positioning to track 
objects and devices? Also, can we do it fast enough to provide a near-realtime
experience?

Research shows that the answer to both the questions is in the affirmative
and papers as far back as 2000 \cite{RADAR} have attempted to achieve this 

\section{Progression of Ideas}

The progression of ideas in this field spans a decade of intensive research. 
However, the central themes can be easily summarized:

\begin{itemize}
\item Initial seminal work was done on indoor location and tracking based on RF signal strength data.\cite{RADAR}
\item The work was improved with the aid of more powerful filters and better algorithms and was applied to different signal sources such as Wifi and Bluetooth.\cite{Kotanen}\cite{SpotON}
\item Analysis of Wifi signals indicated a strong effect of orientation on the sample data. This observation was incorporated into various systems to improve indoor localization accuracy.\cite{Ladd}\cite{King}
\item The advent of Microelectromechanical Systems (MEMS) allowed the development of compact accelerometers on chips and subsequently gyroscopes have also been created. These were sufficient for the development of portable, low cost, Inertial Navigation Systems (INS). An attempt was made to merge the benefits of an INS with a Wifi based approach to indoor localization.\cite{Evennou}\cite{Wang}
\item Particle filters were explored for indoor localization.\cite{Arulampalam}\cite{Ristic}\cite{Widyawan}
\item Implementation of Wifi RSSI based indoor positioning and tracking systems on mobile devices start to appear.\cite{Redpin}
\end{itemize}

Subsequent sections of this chapter describe the papers and their important
ideas and limitations in greater detail in a chronological format.

\section{Early work (2000-2003)}

Bahl and Padmanabhan\cite{RADAR} (2000) proposed RADAR as the first system to utilize RF signals to achieve indoor positioning using a PC. They used a WaveLAN NIC to get signal strength and SNR values from a received and from statistical data collected they were able to carry out indoor location and tracking. Their work was seminal and formed the basis of many indoor localization methods. The central ideas in their paper were twofold:

\begin{enumerate}
\item Treat the problem as a signal processing problem that aims to compensate for free space path loss and other multipath and obstructive effects of the RF source signals by using a propagation model, thus allowing the use of trilateration for positioning. OR
\item Treat the problem as a nearest neighbor problem for a set of predefined signal samples associated with fixed locations in the target area. This was called the location fingerprinting method.
\end{enumerate}

These ideas were central to the research carried out over the next few years.

Kotanen, Hannikainen, Leppakoski et al\cite{Kotanen} (2003) took forward RADAR by exploring alternative radio sources and produced an indoor positioning system using Bluetooth devices as their radio source. Their major contribution was the use of Kalman Filters and other signal processing techniques to improve the accuracy of the location estimate. The major limitation of this method for achieving the stated objectives is that Bluetooth capable “anchor points” are not deployed in sufficiently large numbers and a Bluetooth based indoor location technology would require additional hardware deployment and does not have sufficient range to justify the extra cost and complexity of the solution.

Other systems created during this period include SpotOn\cite{SpotON} by Intel Research which took indoor localization to 3 dimensions using RFID tags. All systems in this period had significant accuracy limitations with best case accuracies 
approaching $3 m$.

\section{Wifi Positioning Algorithm Maturation (2004-2006)}

2004 was a year when indoor positioning really started to generate research interest. The primary progress during this period was the analysis of signal characteristics of IEEE 802.11 systems in the 2.4 GHz band as well as the maturation of approaches to positioning based on probabilistic and fuzzy models.

\subsection{Analysis of the properties of Wifi RSSI}

Kaemarungsi and Krishnamurthy\cite{KStats} (2004)  did a thorough study of the properties exhibited by IEEE 802.11 signals in the 2.4 GHz ISM band as they apply to a real world positioning scenario in this paper. Their major contribution was the statistical analysis of Received Signal Strength datasets. Kaemarungsi and Krishnamurthy first collected data from a single access point to determine the impact of the presence of a user as well as the statistical properties of the dataset (the statistical distribution, the signal stationarity, the variation of the signal at different times of the day etc.). After a thorough analysis, they extended their analysis to datasets generated from multiple access points to determine the degree of independence of signals from the different access points and the similarity of the statistical properties of the access points in a multi-access point scenario. The final part of their work dealt with the differences between RSS fingerprints of two locations to identify commonalities and differences between fingerprints. The major results of their work are summarized in Table \ref{tbl:RSSStats}.

\begin{table}
\centering
\begin{tabular}{p{2in} p{3in}}
\hline
\hline
Statistical Property & Variations \\
\hline

Effect of User’s presence &
Upto 5 dBm variation in the properties \\

Effect of User’s body on Standard Deviation &
Increases from 0.68 dBm to 3.00 dBm \\

Effect of User’s orientation &
Deviations of upto 10 dBm including complete loss of signal strength for Line of Sight obstruction in low signal strength scenarios \\

Statistical Distribution &
Although other published results claim a lognormal distribution of RSS values they fail to mention the presence or absence of a user. As per the authors, in the presence of a user, the lognormal distribution is violated with no clear distribution fit. \\

Standard Deviation &
Except when a user’s orientation blocks an Access Point, the standard deviation values at a particular point are relatively stable. \\

Stationarity of the RSS &
The RSS distribution was found to be fairly stable and exhibited ergodic properties at small time scales but stationarity was violated over longer time scales of the order of hours due to environmental changes. \\

RSS Correlation between APs &
Fairly uncorrelated, nearly independent \\

Interference from multiple APs on same channel & 
Nearly independent on account of MAC layer avoiding simultaneous broadcasts \\

Clustering of RSS values &
RSS values corresponding to different APs show significant clustering behaviour around radio sources and may thus be used with a discriminant to distinguish between locations. Also, the number of distinct tuples for each location are fewer than the number of samples. \\

\hline
\end{tabular}
\caption{Kaemarungsi and Krishnamurthy Analysis Results of RSS datasets (summarized from \cite{KStats})\label{tbl:RSSStats}}
\end{table}

These results are very important as subsequent systems have failed to take the results of this paper into account while designing their algorithms. Their results as well as the groundwork done as part of this thesis forms a significant informing factor for choosing appropriate algorithms for the problem.


\subsection{Orientation Augmented Positioning and Tracking Systems}

Ladd et al (2003, 2005)\cite{Ladd}  took up a different approach to the problem of indoor localization and tracking – they preferred to use the probabilistic viewpoint and take an approach keeping in mind its practical application in robotics. Their paper, published in 2003 as part of IEEE MOBICOM and later in a revised form in 2005 by Springer used a fairly advanced probabilistic model to predict the position of a robot. They used a Bayesian Inference Algorithm to generate probability values for motion and position. They also attempted to coarsely estimate the orientation of a user based on the sample data provided to compensate for orientation effects mentioned in \cite{KStats}. Their model – though fairly complex and quite accurate – is solely built upon a Wifi sample dataset augmented by rudimentary orientation information (discretized to 8 directions) and assumes a gaussian distribution of Wifi signal strengths around a sample (an assumption which is not valid according to \cite{KStats}). However, its central ideas are excellent and the results were the best among its contemporaneous systems.

In 2006, King et al\cite{King} provided the first description of any system that utilized digital compasses to determine user orientation. Their probabilistic algorithm utilized Bayes’ Rule and an assumed Gaussian distribution of signal strengths around the reference points to generate location candidates. Orientation information from digital compass measurements was used to limit the dataset to only those samples in the training set that had a similar orientation as the current test datapoint. Simple averaging of the resulting most probable data points was returned to the user as the most likely location. Bayes' rule was used for localization and no tracking applications were evaluated.

With the number of systems attempting to provide indoor positioning and tracking solutions ballooning, IEEE published a survey\cite{Survey2007} by Hui Liu et al which compared the accuracies and features of the variety of indoor positioning systems. Interested readers may refer to the survey paper for a huge list of positioning and tracking systems and a concise description of the ideas and methodologies behind them as well as a comparative analysis of their features. Such comparisons have been omitted here for the sake of brevity.

\section{INS and Wifi Integration (2006-2008)}

The location system that I found the most interesting and suitable for application to the task at hand was the integration of an Intertial Navigation System (INS) with Wifi measurements done by Frederic Evennou and Francois Marx of the R\&D division of TECH/IDEA, France Telecom. Their paper \cite{Evennou} attempted to fuse an inertial navigation system using  accelerometer and gyroscope data with Wifi location fingerprints for continuous tracking of a user. Their reported accuracy of $1.53 m$ was significantly better than that of pure Wifi location fingerprinting solutions reported previously. However, in their work, they reported that the particle filter used was extremely intensive computationally and was thus not suitable for implementation on mobile devices. Additionally, they used the log-normal approximation relationship for signal strength versus distance - this relationship has been shown to not be valid for indoor scenarios by work done by Kaemarungsi \cite{KStats}. This is apparently a major factor in degrading their results. This led them to believe that the algorithm was not suitable for implementation on a mobile device. To quote:

\begin{quote}
Due to the large number of particles, the
algorithm is too complex to be implemented on handheld
devices. A way to cut down this number of particles must be
found.\cite{Evennou}
\end{quote}

Another major system of this kind was published by Wang et al (2007)\cite{Wang} which fused Wifi Received Signal Strength Information (RSSI) with a step counting algorithm based on \cite{Ladetto} and map information to develop a pedestrian tracking algorithm based on particle filters. Their system was reported to have a mean error of $4.3 m$ with a standard deviation in error of $2.8 m$ in comparison to a pure KNN based system with a mean error of $6.44 m$ and a standard deviation of $6.84 m$. Though the reported accuracy of their system was poor, the paper had a number of interesting ideas.

To the best of the author's knowledge, none of these systems were ever implemented on a mobile device.

\section{Smartphone based Indoor Positioning and Tracking Systems (2008-2011)}

Not many examples of smartphone based indoor tracking systems are found. Redpin\cite{Redpin} was an implementation of a Wifi based room level accurate system on a Nokia N95. The implementation was probably finished around 2008. Subsequent work has produced an iPhone and a slightly unstable Android client for the Redpin server based system - both of which are available in the public subversion repository of the project. Work on this system is ongoing and is concentrating on Wifi as its primary source of positioning information.

Au \cite{Anthea} implemented a Wifi based compressive sensing method for indoor positioning and tracking as part of a Masters thesis in 2010. A simple dead reckoning scheme with error correction based on multiple mobile devices on the same pedestrian has been proposed by \cite{NUSJin} in March 2011. Lukianto, Honniger and Sternberg\cite{Lukianto} claim to be developing an implementation on a Nokia N900 (released 2010) but have not published any algorithms or results. (Note: the author was unaware of these systems at the time of system design and implementation given that they are so recent in nature).


\section{Research Gaps}

Based on a review of a wide variety of papers besides those mentioned in the previous section, the following issues have been identified:

\begin{itemize}
\item Many systems make assumptions of log-normal decay of signals strengths from Wireless Access Points. Such assumptions are not valid as per the research done in \cite{KStats} and verified as part of the groundwork described in Chapter \ref{chap:groundwork}.
\item A decade of research on Wifi assisted indoor positioning system has not provided a satisfactory increase in accuracy of positioning using Wifi data. Even the best systems report errors of the order of $2 m$ with a very high standard deviations.
\item The lower limit of resolution for a Wifi assisted system is 1 wavelength. For systems based on the $2.4 GHz$ IEEE 802.11 standard, this limit is approximately $0.125 m$ in the absence of noise. However, given the amount of noise incurred in the Wifi Signal Strengths (as evaluated in Chapter \ref{chap:groundwork}), it is unlikely that this limit will ever be reached and we shall have to be content with systems that give at best error bounds of around 10x the theoretical maximum resolution.
\item Very few systems appreciate the enormous informative content present in maps and the possibility of using map information for continuous error correction in systems. 
\item None of the dead reckoning solutions being implemented on mobile devices are aiming to use map information in any way to aid tracking.
\item Nearly all the indoor positioning systems were aiming to use Wifi data in some way to decrease the error of their systems. However, this may not always be appropriate.
\end{itemize}

The proposed work aims to improve upon prior work in this area and demonstrate the implementation of indoor positioning systems using a modern Android based smartphone. The discussion of the proposed algorithms is in Chapter \ref{chap:proposed_method}.



