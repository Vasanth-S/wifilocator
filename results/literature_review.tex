\chapter{Literature Review\label{chap:literature_review}}

\section{Location! Location! Location!}

Knowing and finding where we are or where we should be is a very frequent 
task that is performed by us multiple times a day. We solve this problem 
in a variety of ways:

\begin{enumerate}
\item Use our prior knowledge and mental map of our surroundings
\item Ask around
\end{enumerate}

\section{Indoor Positioning}

Indoor positioning has been a topic of active research for the past decade with
the first research system using RF signals for distance estimation and indoor
positioning being produced by Microsoft Research (RADAR, 2000)\cite{RADAR}. Of
course, systems like Active Badge\cite{ActiveBadge} have been developed as far
back as 1992 to locate targets indoors but they have shown limited utility on
account of the requirement to deploy specialized sensors for detecting the radio
tags deployed within the system.

Since RADAR, a number of improvements have been made in the core algorithms and
technologies for positioning. Over time, indoor positioning has progressed from
simply using RF signals to using ubiquitous 802.11 Access Points as radio
sources. With the development of UWB (Ultra Wide Band) technology, very fine
grained ranging and tracking results have been obtained over target distances of
the order of a few hundred feet from the radio source. 

However, the primary scale of interest for commercial exploitation of indoor
positioning and tracking is of the order of the size of warehouses and malls
(approximately of the order of $2500 m^2$) with an accuracy that is preferably
of the order of a few meters. The target applications for such a scenario are
inventory location and tracking; location based services and personalized offers
from businesses based on physical proximity to their shops or other points of
sale. The ultimate target to be achieved in this line of research can be
paraphrased as:

\begin{quote}
Create a low complexity, low cost indoor positioning system that is
maintainable, reasonably accurate and highly scalable.
\end{quote}

UWB, unfortunately, does not scale easily to such large spaces without requiring
that a significant number of sensors and repeaters be installed in the target
positioning area specifically for this purpose. It is also an expensive
technology. On the other hand, IEEE 802.11 AP based positioning has low
complexity and cost and it has a large installed base, thus making it attractive
for positioning purposes as a source of radio signals.

Thus it is interesting to explore the limitations of accuracy, scalability and
maintainability of indoor positioning systems that are based on IEEE 802.11
Access Points.


\section{Indoor Tracking}

Once having worked with indoor positioning systems, it is but natural to ask:
Can we extend all the technology developed for indoor positioning to track 
objects and devices? Also, can we do it fast enough to provide a near-realtime
experience?

Research shows that the answer to both the questions is in the affirmative
and papers as far back as 2000 \cite{RADAR} have attempted to achieve this 


\section{Research gaps}

