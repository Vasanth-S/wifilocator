\chapter{Introduction}

The world is an exciting place today. We are surrounded by technology that has
woven itself seamlessly into our lives. Today, many of us see the time on our
mobile phones, connect with friends and family through mobile versions of
popular social networks accessed through our smartphones. We talk to people
across the globe using that same mobile device and the smartphone doubles up as
a gaming console and eBook reader. The availability of a GPS sensor on 
smartphones has been a growing trend for the past few years and is now nearly
a de facto standard. Implementation of algorithms produced by researchers
over the past two decades on smartphones has resulted in tracking capability
that is quite accurate and suitable for navigation at outdoor locations. 
In fact, latest generation devices like the Samsung Nexus S even ship 
with a beta mobile application that doubles up as a standard car based GPS 
navigation device. So, while engineers at companies are working furiously 
to solve technical challenges for accurate tracking and navigation outdoors,
a different problem looms on the horizon: 

\begin{quote}
How do we take such tracking and navigational aids for mobiles indoors where GPS
based systems are virtually useless?
\end{quote}

This thesis will analyze this question and show multiple systems that 
provide indoor positioning and tracking using not much more than a
cutting edge smartphone. But before moving further, let me state my reasons
for picking this up as my dissertation proposal.


\section{Motivation}

For my dissertation, I wished to take up a topic that was current,
technologically intensive and had a broad scope of application. I had prior
exposure to location based services and map matching algorithms via my seminar
report\cite{SEMINAR} in 2010 and I was firmly convinced that Location Based
Services are going to play a very valuable part in the evolving future of mobile
devices. To enable a future that has relevant location based content, 
reliability of location of the user - both indoors and outdoors is a high 
priority. The outdoor location and tracking problem was being solved 
by using GPS, map matching and other filtering techniques. Long distance 
navigational aids were commercially available and the field is rather mature.
In contrast, the field of indoor positioning was highly fragmented, 
technologically diverse with no clear winners and few commercial offerings. 
The area was thus ripe for research.

\section{Potential Applications}

There are numerous potential applications for research in this area. Some are
obvious - Indoor navigation and tracking, applications as navigational aids for
the differently abled, and even location based services that aim to improve
customer experiences in malls and shopping complexes. However, some of 
the more interesting applications of this technology are in fields that 
are just opening up to research and implementation. 

\subsection{Navigational aids for the abled and differently abled}

Having a stable, highly accurate tracking solution is a prerequisite to 
provide navigational aids in unfamiliar surroundings like a customer at 
a mall or inside a huge departmental store. Visitors to huge office complexes
could also benefit with such navigational aids. The biggest beneficiaries
of such technology are the differently abled - who will be able to 
leverage such tools in coordination with highly mature devices customized 
to their needs so that they can move in indoor surroundings with just as 
much confidence as they do while using other navigational aids like GPS 
outdoors.


\subsection{Inventory tracking and retreival}

There are a number of large industries (eg. shipping industry, online ecommerce
vendors, private couriers, wholesalers etc.) that invest heavily in large
warehouses. One of their primary concerns is inventory management and retreival.
Most of the companies in these industries use some form of inventory tracking 
software that tracks what is present in the warehouses. They also evolve
conventions for storage that allow for easy retrieval of inventory. An indoor
tracking and navigation solution integrated with the inventory system can help 
workers easily and efficiently locate objects present in their warehouses and 
reduce incidents of misplacements or loss.

\subsection{Augmented Reality and Reality Fusion Gaming}

Among all the applications of indoor tracking - this is one that holds the 
greatest excitement.Augmented reality can be defined concisely as:
\begin{quote}
A technology that superimposes a computer-generated image on a user's view of
the real world, thus providing a composite view. (Adapted from \cite{SciDict})
\end{quote}

The field of AR is not new, however, technology has advanced to the point that
AR applications can now be built to run on smartphones while providing
exceptional engagement and interactivity. Researchers at Georgia Tech University
demonstrated such integration through a two player interactive AR game running
on an Android device at the 2010 Uplinq conference. The game was playable
anywhere provided a simple mat denoting the gaming area was placed on the floor.
The virtual characters (or sprites) were then created and layered on a camera
view of the mat. The two players could interact with the characters virtually
and their view of the sprites was altered in near real-time as they moved around
the mat. Another major application of AR to mobile devices has been shown by 
Layar wherein they overlay 3D models and informative content directly onto 
a user's view of a location. 

Integration of a stable indoor tracking solution with such AR applications will
allow the development of a new class of applications where a user's action and
location in real-life will correspond to a sprite's action in the virtual world
thus opening up arenas for AR based first person shooters etc. The possibilities
opened up by such technology are immense and limited only by imagination.

%\missingfigure{Show a typical AR image (sourced and cited)}

\section{Problem Statement\label{sec:problem_description}}
\todo{Please review}
All the applications mentioned above put different constraints of accuracy,
deployability, mobility and scalability on the tracking algorithm. We
concentrate on a specific type of the indoor tracking problem and thus limit
the scope of the thesis to the well defined problem statement below:

\begin{quote}
Develop and characterize a smartphone based particle filter sensor fusion
approach for online indoor tracking of pedestrians. 
\end{quote}


\section{Organization of the Thesis}

The thesis is split into various chapters for the convenience of the reader.

\begin{itemize}
\item Chapter \ref{chap:literature_review} details the flow of ideas and prior
work by researchers in this area.

\item Chapter \ref{chap:proposed_method} specifies the specific methods proposed
by the author for implementation of an indoor tracking system. The methods
detail the modifications proposed for implementation on a mobile device as well
as the choices guiding the approach.

\item Chapter \ref{chap:implementation_details} discusses the implementation
details of the algorithms on an Android device and explains the architectural
organization of the entire codebase.

\item Chapter \ref{chap:groundwork} specifies the field work that was done to
characterize the Microelectromechanical Systems (MEMS) sensors and Wifi receiver on the smartphone and includes
the results of a project that implemented an indoor positioning system based on
the received signal strengths from different Wifi Access Points. This work was
done to determine a choice of suitable parameters for the algorithms specified
in Chapter \ref{chap:proposed_method}.

\item Chapter \ref{chap:results} discusses the performance of various components
of the solution and shows the performance of the final proposed method
contrasted with a naive dead reckoning approach and a Wifi based indoor tracking
approach.

\item Chapter \ref{chap:summary} finally summarizes the experiences derived from
the thesis and describes scope for future work in this area.

\item Additional test results which were
too voluminous to include in the main text of this thesis are present in 
the Appendix.

\item The papers and other sources of information used in this thesis are listed
in the bibliography at the end of this thesis.
\end{itemize}
