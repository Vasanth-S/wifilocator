\chapter[Proposed Work]{Particle filter Based Map and Sensor Fusion technique for Indoor Tracking 
using an Android Smartphone\label{chap:proposed_method}}

\section{Relevant Prior Work}
As described in Chapter \ref{chap:literature_review} (Literature Review),
the pedestrian navigation system described in \cite{Wang} 
shows the most promise for accurate tracking of subjects in indoor tracking scenarios. 
The results of \cite{Evennou} which implements a similar system integrating an INS 
with Wifi data are very encouraging. However, we are cautioned 
against the computational complexity of a particle filter based solution and its attendant
implications while implementing the same on a mobile device by Evennou and Marx\cite{Evennou}. As we shall later
see, these cautionary words are well founded.

Our proposed method thus builds upon the work of Evennou and Marx\cite{Evennou}
and that of Wang et al\cite{Wang}. However many modifications specific to
implementation on a resource constrained device like an Android smartphone are
made. The proposed device for implementation is the Samsung I9020 (also called
the Samsung Nexus S which is co-branded with Google).

%Specifically, we discard the use of Wifi information for the central tracking
%algorithm and instead use it only in the error recovery phase. The justifications
%for these modifications are made in the accompanying text.
%modifications are proposed and implemented to take into account the fact 
%that our implementation system is an Android smartphone with a number of 
%sensors and limited processing capabilities. 

%The device that is the test-bed for implementation is the Samsung Nexus S. 
%The onboard accelerometer, magnetometer and gyroscope of the Nexus S are used 
%as inertial navigation sensors. The algorithm is then analyzed with different 
%parameters and paths and is compared and contrasted with two other tracking 
%approaches implemented on the same device - one a simple uncorrected dead
%reckoning implementation and the other a pure Wifi based based tracking system.
%These results are summarized in Chapter \ref{chap:results}.

\section{System Inputs\label{sec:system_inputs}}

Given the choice of hardware, we are constrained to use only specific 
data sources for our algorithm. 
The sensors available on a typical Android Smartphone are:
\begin{enumerate}
\item A 3-axis accelerometer for measuring acceleration of the device.
\item A 3-axis magnetometer for measuring the surrounding magnetic field.
\item A gyroscope for providing angular velocity information.
\item A camera for providing visual information.
\item A Wifi Network Card with the ability to provide measurements of the received signal strengths. (RSSI)
\end{enumerate}

Besides the sensors mentioned above, additional sources of information 
that we can make available to the system are:

\begin{enumerate}
\item Map information detailing the test environment.
\item A site survey of the test environment which provides Wifi location fingerprints augmented by 
    orientation information of the reciever when the fingerprint was taken.
\end{enumerate}

Characterizations of these data sources as well
as other relevant parameters will be done as ground-work in 
Chapter \ref{chap:groundwork}.

\section{Device Limitations\label{sec:device_limitations}}

Being a resource constrained device, a number of limitations are imposed on 
software for these systems. Some of the ones most affecting the implementation
are:

\begin{itemize}
\item Applications are restricted to a maximum memory utilization of 16 MB on 
    older devices and newer devices get limits based on the size of their 
    RAM. The Nexus S has a maximum limit of 32 MB, other devices usually have
    lower limits of 24 MB or 16 MB.
\item The CPU usage of the application has to be controlled. Applications 
    that use the CPU intensively for more than a few seconds are detected as 
    misbehaving applications and the user is prompted to kill them.
\item Sensor events are dropped if the event delivery thread is busy processing
    the last delivered event. This puts significant limits on the 
    amount of inter sample processing that may be performed while 
\end{itemize}


\section{Background on Particle Filters}
\todo{What/How much to write here and where to put in the thesis?}
Particle filters belong to a class of Sequential Monte Carlo algorithms 
that depend essentially approximating a probability distribution function based
on a Monte Carlo simulation of the system using observation samples. Particle 
filters have been known to be very useful for tracking in systems where 
probability distribution functions are either not completely known or hard 
to sample. 

In this case, our problem of indoor tracking is essentially the 
determination of a latent variable $x_t$ which represents the position of the 
user after we receive a series of observations ($o_0 \dots o_t$) from the 
sensors on the Android device. Effectively, we wish to approximate the the 
probablity distribution $p(x_t|o_t,o_{t-1}\dots,o_0)$ at any time instant $t$
so that we may determine the position of the device using the expectation 
function $E[x]$ on the probability distribution $p(x_t|o_t,o_{t-1}\dots,o_0)$
and we wish to use a particle filter to help us do so.

Particle filters are a huge topic in and of themselves. Thus, in the interests
of brevity, the reader is referred to the excellent expositions of particle 
filters for tracking applications in \cite{Ristic} and to \cite{Arulampalam}
for particle filters in the context of non-linear, non-gaussian Bayesian
tracking.


\section{Proposed Method}

Keeping the availability of the inputs described above as well as the device 
limitations described in Section \ref{sec:device_limitations} in mind, a less 
computationally intensive method is proposed in 
Algorithm \ref{algo:proposed_method}.

\begin{algorithm}
\SetKwInOut{Input}{Input}
\Input{Sensor Events from device sensors}
\KwResult{An estimate of the location of the device at any time through the variable $L$}
\Begin{
    $L \longleftarrow FirstFix()$ \;
    $P \longleftarrow$ $N$ particles around $L$ corrupted by Gaussian Noise\;
    \ForEach{$stepEvent$ from $sensors$}{
        $P' \longleftarrow ApplyDynamicalEquations(P, stepEvent)$\;
        $P' \longleftarrow MapSelect(P', P)$\;
        \If{$P'$ is $\emptyset$}{
            $R \longleftarrow PerfomRecovery(P)$\;
            \If{$R$ is not $\emptyset$}{
                $P \longleftarrow R$
            }
        }{
            $P \longleftarrow P'$\;
        }
        $L \longleftarrow Mean(P)$\;
    }
}
\caption{High Level View of the System\label{algo:proposed_method}}
\end{algorithm}

Algorithm \ref{algo:proposed_method} is very general. Also, though it is 
stated in a sequential manner, the it needs to 
be implemented in an event-driven fashion to allow for online tracking. 
The \textbf{ForEach} loop in the algorithm is essentially an event loop 
that depends on step detection.

Procedures to determine \emph{FirstFix}, detect steps from sensor data, 
\emph{MapSelect}, \emph{ApplyDynamicalEquations}, and \emph{PerfomRecovery} are explained
below.

%\begin{enumerate}
%\item A first-fix position estimate will be taken by means of manual input or 
%    via QRCodes through the smartphone camera.
%\item A step counting approach in accordance with a modified peak and 
%    valley approach and a step stride estimation in accordance with
%    \cite{ADXL202} will be used.
%\item A low complexity particle filter is used to implement the tracking 
%    algorithm. 
%\item Map information is used to provide means for correcting drift in the 
%    simple tracking algorithm.
%\item Additional "hidden" variables are used in the dynamical equations of the 
%    system to compensate for a constant sensor drift and small variations in 
%    the step size of the user. 
%\end{enumerate}

Comparison of the proposed method will be done with 2 other algorithms, also
implemented on the smartphone: 

\begin{enumerate}
\item A simple step based dead reckoning solution that directly uses sensor data.
\item A simple step based dead reckoning solution with a Wifi based nearest neighbour algorithm incorporated as a 
    drift correction method.    
\end{enumerate}

These algorithms are being implemented as control algorithms for comparisons
in order to provide suitable comparative information since maps and test 
environments differ substantially and make comparative analysis difficult.

\section{First Fix\label{sec:first_fix}}

Dead reckoning refers to a method of position determination where the current
location is estimated based on a known starting location and offsets from the
known location measured via instruments.

For any dead reckoning algorithm, it is important to provide a good starting
point. Low error in the starting location contributes to better performance.
Our proposed method is based on the dead reckoning concept and thus requires 
a (reasonably) accurate starting point (also called a first fix).

In order to provide the starting point to the algorithm two approaches are 
proposed:

\begin{enumerate}
\item Directly choose the starting location on a map using the capacitive 
    touchscreen
\item Select the starting location based on QRCode recognition using the onboard
    camera. The QRCodes were printed and pasted at specific locations 
    in the test environment and the user is required to simply take a 
    picture of it using the onboard camera.
\end{enumerate}

\subsection{QR Codes\label{sec:QRCodes}}
QRCodes are a kind of 2 dimensional machine readable code (an example of which
is shown in Figure \ref{fig:sample_qrcode}). The information encoded in this
format can be read off by processing an image acquired from a camera. 

QRCodes can contain a variety of content - 
from URLs to Contact information. They can also contain plain text. 

For this application, we use the plain text information mode of QRCodes 
to provide encoding of a human-readable text representation of map points. 
This human and machine readable representation is achieved with the lightweight
JavaScript Object Notation (JSON) representation. 
Details of the format chosen for this application are
mentioned below:

\begin{figure}
    \centering
    \includegraphics[width=3in]{figures/sample_qrcode}
    \caption{A Sample QRCode\label{fig:sample_qrcode}}
\end{figure}


\subsubsection{QR Code information format}

A typical format used for storing all the relevant information in a QRCode is
shown in Figure \ref{fig:QRCode_info_format}. As is proper for any data 
serialization format, it stores a Version number and a Type field to 
distinguish itself from other data serialization formats. It also includes
critical information about the point itself - the X and Y locations on 
the relevant map as scaled on a 0-1 scale. A full description of the fields
is given in Table \ref{tbl:QRCode_fields_table}.

The fields are descriptive 
and quite human readable. However, because they are in JSON, a basic 
check of well-formedness can be made by programs. The data format retains 
extensibility as it is able to accommodate additional fields which will
be ignored by applications that are not built to expect the presence of those
fields.

\begin{figure}
    \centering
\begin{verbatim}
{
   "Version": 1,
   "Type": "MapPoint",
   "MapURL": "http://www.iitr.ernet.in/path/to/sample/map.png",
   "Scale": 0.010716161,
   "X":0.70625,
   "Y":0.16479166666666667 
   
}
\end{verbatim}
    \caption{A sample data point encoded as text information in a QRCode\label{fig:QRCode_info_format}}
\end{figure}

\begin{table}
\centering
\begin{tabular}{p{1in} p{4in}}
\hline
\hline
Field Name      &       Field Description \\
\hline
Version         & Constant value 1. May be incremented if additional fields are added to the format. \\
Type            & Represents the type of QRCode waypoint sample that was just acquired. Type "MapPoint" indicates that the QRCode represents a point on a Map. Alternative and  additional information can be provided by choosing other Types. For example,  a Type of "WifiAP" could be used to provide location information about  Wifi Access Points in the surroundings. \\
MapURL          & This field always refers to a publicly available map associated with the environment where  the QRCode was pasted. A png map type has been indicated, but any other resource type that  a client is able to handle should be accepted. \\
Scale           & Scale is an optional parameter of a MapPoint. It represents the scaling factor between distances on a map with distances in the real world. It can be omitted if it is expected that the  client will have some other way of figuring out the scale of the map based on information encoded within the map itself. \\
X and Y         & These are values encoded in the real range [0-1] which represent the location of the QRCode on a map referred by it. These values are used for providing a first fix to the  reckoning algorithms.  \\
\hline
\end{tabular}
\caption{Explanation of the fields used in the QRCode Information Format\label{tbl:QRCode_fields_table}}
\end{table}

\section{Noise filtering\label{sec:NoiseClamping}}

It is well known that while measuring real world information, most sensors
 pick up noise from the environment that affects the readings of the 
sensed variables. The MEMS accelerometer present on a typical smartphone is 
no different. 

In general, we assume that the accelerometer sensor noise is below a small 
threshold $T$ when the device is static on a firm surface such as a table. 
A higher threshold $Q$ is required when the device is being kept at the palm of 
a hand because of small noisy variations introduced by the palm itself.

The filtering process uses a simple clamping mechanism. The filter 
rejects a reading of the accelerometer based on the following constraints:

\begin{equation}
NoiseFilter(a_i) =  \begin{cases} 0 & \text{if $|a_i| \le Q$,} \\
                                a_i & \text{otherwise}
                    \end{cases}
\end{equation}

The values of T and Q are determined as part of the groundwork in Chapter 
\ref{chap:groundwork}.

The filtering process is very important for the step counting algorithms
to detect accurate peaks.

\section{Distance Estimation}

The acceleration of the human body as part of the act of walking is small and 
very impulsive in nature. Unfortunately, the linear acceleration values
produced by it are so small that they are virtually indetectible from the 
sensor noise of the MEMS accelerometer of the Android device. However, the 
act of putting a foot on the ground generates an impulse along the Z 
axis of the accelerometer which is clearly distinguishable in the sensor 
readings. Thus, we use the step count method described in
\cite{Wang} and \cite{Ladetto} to detect motion of the user holding the device.

By relating the acceleration impulses to the size of a step, 
an inertial navigation system may be created. An empirical formula 
from \cite{ADXL202} is used to estimate step sizes from the readings sensed by
the accelerometer. Details of all the methods used to achieve the same 
follow:

\subsection{Step detection procedure\label{sec:step_detection}}

\subsubsection{Zero Crossings}

This step counting method is described in \cite{Wang} is based on a similar 
method devised for outdoor pedestrian navigation by \cite{Ladetto}. It is a
simple formulation that counts the number of zero crossings of a 
filtered version of the raw accelerometer signal which represent double the 
number of steps taken. However, this method, when implemented, generates 
step events at zero crossings which correspond to a body in motion.
Since each step has to be associated with an angle of motion, it might be 
advantageous to ensure that step events are associated with points where 
the foot of the user is on the ground, leading to better stability of angle 
readings. Thus, the Peak and Valley hunting method is also proposed. 

\subsubsection{Peak and Valley hunting\label{sec:peak_and_valley}}

The Peak and Valley hunting procedure is an alternate method being
proposed for step detection.
This method will detect the same number of crossings as the zero crossing 
method but will raise the step events whenever the foot of the user 
strikes the ground for the beginning of the next step. 

In this method a 2-state machine is constructed according to Table
\ref{tbl:peak_valley_state_table}. This method outputs a detected step at 
the first positive peak in the accelerometer sensor data that corresponds to 
the beginning of the next step. Also, since it only processes data at 
points where the first derivative of the accelerometer signal is zero and 
the accelerometer signal itself is non-zero, it can reduce computational 
effort for cases where zero crossing would perform considerable testing
on account of being at the value 0, waiting for a transition. Additionally,
the values of $A_{max}$ and $A_{min}$ that will be required later are 
updated only during these intervals, thus saving further computational 
resources compared the Zero Crossing method. The disadvantage of this method
though is the additional lag introduced between a step being taken and its 
event being delivered to the application.

\todo{Improve this area}
\begin{table}[h]\centering
    \begin{tabular}{c p{1in} c p{2.7in}} \hline
    State & Accel. Value     & New State &  Action\\     \hline
    $q_0$ & Positive Peak ($P$) Detected  & $q_1$     & Update $A_{max}$ if peak $P > A_{max}$. Output a step event if $flag = 1$. Reset $flag$ to 0.  \\ 
          & Other values            & $q_0$     & Ignore \\         \hline
    $q_1$ & Negative Trough ($T$) Detected & $q_0$    & Update $A_{min}$ if $T < A_{min}$. Set $flag$ to 1. \\
          & Other values            & $q_1$     & Ignore \\ \hline
    \end{tabular}
    \caption{State table of the step detection state machine\label{tbl:peak_valley_state_table}}
\end{table}

\subsection{Step Size Estimation}

Engineers from Analog Device have published an empirical relationship between 
acceleration values and step size in \cite{ADXL202}:

\begin{equation}\label{eq:step_size}
 Step-size = C \sqrt[4]{A_{max} - A_{min}}
\end{equation}

The constant $C$ is a scaling factor that is used as a constant of proportionality
to scale the step-values to real world distances and $A_{max}$ and $A_{min}$
represent maximum and minimum acceleration values corresponding to the 
peaks and troughs associated with a step. 

This equation is used to obtain an estimate of the step size when a user 
takes a step.

\subsection{Determining the Training Constant}

The training constant for each user was determined experimentally. 
Users were asked to perform a short walk between 2 QRCoded locations
present in a straight line along a corridoor.
Since, the QRCodes represent fixed locations in the real world, the actual 
distance between them can be found using simple Euclidean geometry. From 
simple step counting, we are able to figure out the number of steps taken.
Also, by means of the step size formula mentioned in \eqref{eq:step_size},
we can estimate the distance travelled based purely on the accelerometer
values. We can then find the training constant by simply plugging in 
the values into the following equation:

\begin{equation}
C=\frac{\sqrt{(x_{2}-x_{1})^{2}+(y_{2}-y_{1})^{2}}}{\sum_{n=1}^{stepcount}\sqrt[4]{(A_{max_{i}}-A_{min_{i}})}}
\end{equation}

Here,\\
\begin{tabular}{l l}
$C$                         & is the training constant   \\
$(x_1, y_1), (x_2, y_2)$    & are the anchor point locations provided by the QRCodes \\
$A_{max_{i}}, A_{min_{i}}$  & represent the maximum and minimum \\
                            & acceleration values corresponding to the $i^{th}$ step.\\
$stepcount$                 & The number of steps detected by the algorithm in section \ref{sec:step_detection} \\
\end{tabular}


\section{Dynamical Equations for the System}

With the preliminaries now out of the way, we can develop the dynamical model
of our proposed solution. The basic step update equation for the system 
can be written as:

\begin{equation}\label{eq:dr_eq}
\begin{bmatrix}x_{i+1}\\
y_{i+1}
\end{bmatrix} = \begin{bmatrix}x_{i}\\
y_{i}
\end{bmatrix}  + d{}_{i} \begin{bmatrix}-cos(\theta_{i})\\
sin(\theta_{i})
\end{bmatrix} 
\end{equation}

Here,\\
\begin{tabular}{p{1in} p{4in}}
$x_i, y_i$          &   represent location of the device after the $i^{th}$ step\\
$x_0, y_0$          &   are the first fix values obtained via the methods of Section \ref{sec:first_fix}\\
$d_i$               &   the predicted step distance as per the step estimate equation \eqref{eq:step_size}\\
$\theta_i$          &   the angle associated with the detected step based on a magnetometer reading\\
\end{tabular}

These equations are written assuming a coordinate system for the map where $x$
positive towards right and $y$ positive downwards with $TrueNorth$ of the map
pointing upwards. Effectively, this equation represents a raw dead reckoning
solution. Unfortunately, these equations do nothing to counteract the biggest
disadvantage of dead reckoning: unbounded error growth. The dynamical
representation is also overtly simplistic because it fails to take into account
other issues such as sensor drift and sensor bias. The magnetometer is the
biggest culprit in this regard as it is highly sensitive to stray magnetic
effects in the environment.

\subsection{Modelling Orientation Sensor Noise}

\begin{equation}
\begin{bmatrix}x_{i+1}\\
y_{i+1}
\end{bmatrix} = \begin{bmatrix}x_{i}\\
y_{i}
\end{bmatrix}  + d{}_{i} \begin{bmatrix}-cos(\theta_{i}+\vartheta)\\
sin(\theta_{i}+\vartheta)
\end{bmatrix} 
\end{equation}

We modify the dynamical equation to add an additional parameter $\vartheta$
which is a random variable that represents random variations akin to white noise
in the reading from the true value of the magnetometer. $\theta_i$ now
represents the true angle associated with the step motion.

\subsection{Accounting for Orientation Sensor Bias\label{sec:angle_bias}}

Angular readings from the magnetometer often turn out to be biased from
$TrueNorth$. This bias can creep in due to 2 different reasons - the first being
specific, environmental magnetic fields which distort the actual detection of
$TrueNorth$ in the system and the second being a bias that creeps in due to the
way the user holds the smartphone in the palm of his hand and the offset thus
produced. To take into account such offsets, we modify the dynamical equations
as follows:

\begin{equation}
\begin{bmatrix}x_{i+1}\\
y_{i+1}\\
\theta_{b_{i+1}}
\end{bmatrix} = \begin{bmatrix}x_{i}\\
y_{i}\\
\theta_{b_i}
\end{bmatrix}  + \begin{bmatrix} d{}_{i} & d_i & 1 \end{bmatrix} \begin{bmatrix}-cos(\theta_{i}+\theta_{b_i}+\vartheta)\\
sin(\theta_{i}+\theta_{b_i}+\vartheta)\\
\theta_t\\
\end{bmatrix} 
\end{equation}

\begin{equation}
\theta_t \sim \mathcal{N}(0,\sigma_{angle})
\end{equation}

In this modified version of the dynamical equations, we have added a slowly
varying term $\theta_{b_i}$ that represents an explicit bias in the readings
from the magnetometer at the time of the $i^{th}$. This bias value is updated at
each step by altering it with a small gaussian noise of zero mean an a variance
determined by the characteristics of the magnetometer. The choice of
$\sigma_{angle}$ was made on the basis of the groundwork done. Corrections
for the sensor bias will be made when $MapSelect$ is applied to the particles by
the overall algorithm.

\subsection{Accounting for Varying Step Sizes and Missing Steps\label{sec:step_bias}}

The second issue at hand is step size variation and missing steps. Steps may
be missed by the step detection procedures or phantom steps may be detected
due to sensor noise or other events. Changes in step sizes due to changes in 
footwear or floor material as well as the leads and lags produced due to a 
slightly incorrect calibration constant also contribute to deviations of 
apparent distance travelled versus the apparent distance travelled. To account
for this bias in step size detection, we introduce an additional parameter
$d_{b}$ in the dynamical system that accounts for the step bias. 
Thus, the new equations for the dynamical system are:

\begin{equation}
\begin{bmatrix}x_{i+1}\\
y_{i+1}\\
\theta_{b_{i+1}}\\
d_{b_{i+1}}\\
\end{bmatrix} = \begin{bmatrix}x_{i}\\
y_{i}\\
\theta_{b_i}\\
d_{b_i}
\end{bmatrix}  + \begin{bmatrix}(d{}_{i}+d_{b_i}) & (d{}_{i}+d_{b_i}) & 1 & 1\end{bmatrix} \begin{bmatrix}-cos(\theta_{i}+\theta_{b_i}+\vartheta)\\
sin(\theta_{i}+\theta_{b_i}+\vartheta)\\
\theta_t\\
d_{t}
\end{bmatrix} 
\end{equation}

\begin{equation}
d_{t} \sim \mathcal{N}(0, \sigma_{step})
\end{equation}

In this representation, $d_{b_i}$ is a slowly varying bias variable on the step
size and the gaussian variance used during each update step is $d_t$. The 
determination of the variance of $d_t$ again depends on sensor to sensor and 
is evaluated empirically for our system to be close to $step\_size/25$.

Note that the equations mentioned in Sections \ref{sec:angle_bias} and
\ref{sec:step_bias} will work only if the map information is dense enough to
eventually eliminate groups of particles with incorrect orientations and 
step drifts. Otherwise, the only effect of maintaining the bias variables will 
be to slightly increase the variance of the variables $x_i$ and $y_i$.
An explanation of how these bias variables actually help determine overall
angle and step bias is provided in Section \ref{sec:overall_bias}

\subsection{Ensuring particle diversity}

The presence of the variables $\theta_{t}$ and $d_t$ represents an 
intentional introduction of randomness into the system to ensure 
particle diversity. For example, if two particles start out with the same
state vectors, say 
$ [ x_i y_i \theta_{b_i} d_{b_i} ] $ and 
$ [ x_j y_j \theta_{b_j} d_{b_j} ] $, they will in the immediate 
next time step tend to diverge due to possibly different samples of the 
random variables $\theta_{t}$ and $d_t$. So, 
when evolved over a large number of steps, similar samples will take 
slightly different paths through the state space, thus ensuring particle 
diversity of the particle filter. Note however, that the choice of 
magnitudes of the variance of $\theta_t$ and $d_t$ is critical. If the 
values chosen are too small, states that were possible physically are 
not achieved by the particle filter and if the values chosen are too large,
computational effort will be wasted for unachievable states. For our case 
of a highly resource constrained machine, this is very undesirable for we
wish to maximize the effectiveness of our particles. In this regard, it is 
also very advantageous if we have a highly detailed map of the area for 
unachievable states are quickly pruned and computational effort is 
redirected towards states with more likelihood. The integration of 
map information is described in the subsequent section.

\section{Integrating Map Information}

Map information can be integrated into the system in a number 
of ways. The simplest way to use map information as a selection 
function which is similar to \cite{Wang}:

\begin{equation}\label{eq:select}
MapSelect(p_{i+1}, p_i) = \begin{cases}1 - P_{wall} & \text{if $p_i + \alpha (p_{i+1} - p_i)$ for $\alpha \in [0,1]$} \\
                                                    & \text{does not cross a wall}\\
                                    P_{wall} & \text{otherwise}
                          \end{cases}
\end{equation}

Here,\\
\begin{tabular}{p{1.5in} p{3.5in}}
$p_{i+1}$ and $p_i$ &   are tuples corresponding to map locations ($x_{i+1}$, $y_{i+1}$) and ($x_i$, $y_i$) respectively. \\
$P_{wall}$ &   represents the probability of selection of the particle if it crosses a wall. \\
$\alpha$ & is a parametric variable used to represent a line between the two points $p_{i+1}$ and $p_i$ \\
\end{tabular}

Wang et all\cite{Wang} suggest that a $P_{wall}$ of 0 be used as such a motion
is impossible. Though simplistic, this approach works well by swiftly removing
unreachable particles. It is very light computationally too. However, given our
very limited budget for particles, it might not be prudent to remove a particle
simply because it was crossing a wall very close to a door. Thus, more advanced
map integration techniques may well be envisaged with $P_{wall}$ varying based
on the distance of the intersection point from the closest door in the wall. 

\section{Determination of overall orientation and step biases\label{sec:overall_bias}}

The integration of map information with the dynamical equations of the system 
via a selection function $MapSelect$ allows us to actually determine 
the extent of orientation and step bias. The idea behind it is simple. 
However, before explaining it, let us define the following two terms:
\begin{equation}
\displaystyle
\theta_b = \frac{1}{N}\sum\limits_{i} \theta_{b_i}
\end{equation}

\begin{equation}
\displaystyle
d_b = \frac{1}{N}\sum\limits_{i} d_{b_i}
\end{equation}

Here,\\
\begin{tabular}{p{1.5in} p{3.5in}}
N & is the total number of live particles in the system \\
$d_{b_i}$ and $\theta_{b_i}$ & retain their meanings from the previous sections \\
$\theta_b$ & represents overall system orientation bias at the $i^{th}$ stage \\
$d_b$ & represents overall system step bias at the $i^{th}$ stage \\
\end{tabular}

When no walls exist in the vicinity of the particles, map information is sparse
and few particles are eliminated because of it. Thus, we have insufficient 
feedback from our system to determine the value of orientation bias ($\theta_b$)
as all positions are equally valid. However, in a confined space like a 
doorway or a corridoor, there exist only a few valid range of angles over 
a sequence of steps. Thus
errant particles will be swiftly rejected by $MapSelect$ and we will 
get a highly accurate estimate of the degree of orientation bias present 
in our system.

The justification for step bias ($d_b$) lies similarly in the fact that while
moving along corridoors, we will have accumulated
a certain degree of step drift. This drift ensures that we are slightly 
uncertain of our exact position along the corridoor after walking through it for a long time
as we have not received any feedback from $MapSelect$ along our direction of motion. 
Thus,
some particles representing our location will be further along the corridoor than others and we have no way 
to choose between them other than by treating them at par. However, 
whenever we make a sharp turn on the map, the presence of walls and other 
obstacles immediately before and after the junctions on the map will cause 
a number of particles to be rejected by $MapSelect$. The values of $d_{b_i}$
remaining represent the net bias in the step size of the system. Thus,
their mean $d_b$ is a good representation of the step bias of the system.

\section{Recovering from particle insufficiency}

Particle insufficiency is a weakness of particle filters. It arises if 
no valid successor states $p_{i+1}$ can be found from the current set of 
particles $p_i$ when the observation $o_i$ is taken into account and applied 
to the dynamical equations of the system. (That is, all successor states 
$p_{i+1}$, have $MapSelect(p_{i+1}, p_i) = 0$). This possibility is 
further accentuated in resource constrained systems like ours which 
have a hard limit on the maximum number of particles that can be 
updated between two steps (since the algorithm is working in an online
fashion with a location estimate being available after every step).

The insufficiency of particles can be handled by a number of ways. 
The simplest one is: increase the number of particles being updated at each 
step and ensure enough particle diversity so that you always have 
a few particles that survive incorrect decisions of the particle filter. 

Widyawan \cite{Widyawan} suggests an alternative method: maintaining particle
history information so that an incorrect decision of the particle filter can be
rolled back a few steps in the past and an alternative decision can be made that
does not cause the particle filter to suffer particle insufficiency. However,
the method suggested is expensive in terms of memory usage and computational
cost.

The recovery algorithm ($PerformRecovery$) being proposed is stated in 
Algorithm \ref{algo:recovery}.

\begin{algorithm}
\dontprintsemicolon
\linesnumbered
\SetKw{Break}{break}
\SetKwInOut{Input}{Input}
\Input{A set of particles $P$ such that $ApplyDynamicalEquations(P)$ yields $\emptyset$, $\sigma_X$ and $\sigma_Y$ are the positioning errors of the Wifi based positioning system}
\KwResult{A new set of particles $R$ which approximate the location of the device or $\emptyset$}
\Begin{
    \emph{Try to recover by finding particles in the surroundings by random sampling}\;
    $retryCount \longleftarrow 0$\;
    \While{$retryCount < 2$}{
        \For{$j$ in $[1..N]$}{
            $r_j \longleftarrow (\mathcal{N}(0, \sigma_X), \mathcal{N}(0, \sigma_Y))$\;
            $selectParticle \longleftarrow false$\;
            \ForEach{$p_i \in P$}{
                \If{$MapSelect(r_j, p_i)$}{
                    $selectParticle \longleftarrow true$\;
                    \Break\;
                }
            }
            \If{$selectParticle$ is $true$}{
                $R \longleftarrow R \cup \{ r_j \}$\;
            }
        }
        $R' \longleftarrow ApplyDynamicalEquations(R)$\;
        \If{$R'$ is not $\emptyset$}{
            \Return $R$\;
        }
        $retryCount \longleftarrow retryCount + 1$\;
    }
    \emph{We have failed to find suitable particles by random sampling, fall back on Wifi}\;
    $wifiLocation \longleftarrow PerformWifiPositioning()$\;
    \For{$j$ in $[1..N]$}{
        \emph{Corrupt the wifi location with gaussian noise to approximate the error}\;
        $r_j \longleftarrow wifiLocation + (\mathcal{N}(0, \sigma_X), \mathcal{N}(0, \sigma_Y))$\;
        $R \longleftarrow R \cup \{ r_j \}$\;
    }
    $R' \longleftarrow ApplyDynamicalEquations(R)$\;
    \If{$R'$ is not $\emptyset$}{
        \Return $R$\;
    }{
        \emph{We've failed to find suitable particles, return failure and handle it by skipping the step information.}\;
        \Return{$\emptyset$}\;
    }
}
\caption{Recovery algorithm for particle insufficiency\label{algo:recovery}}
\end{algorithm}

There are a few interesting things to note about the recovery algorithm.
The first being that it retries finding suitable particles. This is an 
optimization as we don't want to spend too much effort at recovering 
unless we really have to since the inner algorithm has an O($N \times |P|$) 
complexity. In case we fail to recover directly from random sampling 
around the location, we retry with Wifi Positioning data. If that fails 
too, then we return an empty set. The system handles this failure by skipping 
the step event (discarding it as possibly invalid sensor readings).

\section{Summary}

The above algorithms represent a lower complexity particle filter that 
is proposed to be suitable for implementation on resource constrained 
smartphones. The performance of the proposed methods is analyzed in 
Chapter \ref{chap:results}.

