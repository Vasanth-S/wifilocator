\chapter{Proposed Method\label{chap:proposed_method}}
The particle filter approach described in [ped. Nav.] is adapted to and
implemented on the Android smartphone. The onboard accelerometer and
magnetometer of the Nexus S are used as inertial navigation sensors.
Improvements are made to the algorithm. The performance of the particle filter
is analyzed with different parameters and approaches:

\section{Dead reckoning with step counts}

The acceleration of the human body as part of the act of walking is small and 
very impulsive in nature. The motion of the torso of the body is governed 
mostly by inertia. Unfortunately, the impulsive acceleration values are small 
and lie buried in sensor noise. They are virtually indetectible using the MEMS 
accelerometer supplied with the android device - Nexus S.

To overcome this difficulty, the step count method of [ped. Nav.] is adopted. 
Under the assumption that step size varies very little over the course of 
motion of the subject, an inertial navigation system may be created.

\subsection{Step counting method}

The step counting method is described in [ped. Nav.]. However, we take a
slightly modified approach.  

\subsection{Noise clamping\label{sec:NoiseClamping}}

The MEMS accelerometer sensor on the smartphone has sensor noise present close
to zero. The noise level for the accelerometer was determined to be always less
than $0.6 m/s^2$. To allow for a reasonable noise margin and provide sufficient
cushion for additional noise introduced due to the dynamic nature of walking, we
choose a threshold of $1.3 m/s^2$. If the absolute value of the Z-axis
acceleration sample is less than this threshold, then the sample will be clamped
to zero. 

As has been determined experimentally, steps taken by a human usually produce
a pair of spikes - one positive and one negative with magnitude around $2 m/s^2$.
By choosing this threshold value, we give maximal noise margin for step detection.

See figure [...] for a comparative estimation of step values versus sensor noise.

\subsection{Step size detection procedure}

\subsubsection{Zero Crossings}

To detect actual steps taken by the subject holding the device, [ped. Nav.] 
suggests using zero crossings. However, in the sensor data collected, a number
of spurious zero crossings exist (primarily due to sensor noise). However, 
even after sensor noise is clamped as per section \ref{sec:NoiseClamping}, 
spurious zero crossings often arise due to variable motion of the subject.

\subsubsection{Peak and Valley hunting}
Peak and Valley hunting procedure (better than zero crossings) for step detection.
Robust to unexpected values.

To have robust detection of steps, a state machine is maintained. The state 
machine detects a peak and then waits for a trough. Subsequent

An internal state machine is used. The state machine has 2 states and a comparison
is made between $A_{max}$ or $A_{min}$ and the sample value that forms the peak/trough
whenever state transitions occur.

\begin{table}[h]\centering
    \caption{State table of the step detection state machine}
    \begin{tabular}{cccc} \hline
    State & Accelerometer Value     & New State &  Action\\     \hline
    $q_0$ & Positive Peak Detected  & $q_1$     & Update $A_{max}$ if peak is of larger magnitude \\
          & Other values            & $q_0$     & Ignore \\         \hline
    $q_1$ & Negative Trough Detected & $q_0$    & Update $A_{min}$ if trough is negative \\ 
          &                         &           & and of larger magnitude than $A_{min}$ \\
          & Other values            & $q_1$     & Ignore \\ \hline
    \end{tabular}
\end{table}
\subsection{Step Size Estimation}

Reference [A. Engineers] provides this empirical relationship between acceleration
values and step size.

\begin{equation}
 Step-size = C \sqrt[4]{A_{max} - A_{min}}
\end{equation}

The constant $C$ is a scaling factor that is used as a constant of proportionality
to scale the step-values to real world distances.

\section{Determining the Training Constant}
A simple procedure was followed to determine the training constant for a user.
The user was asked to locate himself using a QRCode before starting out along
a straight line along a corridoor

\section{Sensor fusion using Particle Filters}

A background on particle filters has already been given.

\subsection{Dynamical equations for system}

The dynamical equations that govern the system are:



\subsection{Accelerometer}


\subsection{Orientation}

\subsection{Camera info}
For first fix, we use QR codes. They are used whenever subject changes floors too.
The QR codes provide information about the map of the floor, the current location
and any additional information that is required by the tracking algorithm.

\subsection{Map Information}



\subsection{Wifi Information}

\section{Accounting for orientation bias and noise}

\section{Accounting for varying step sizes}

\section{Barometer Information}

