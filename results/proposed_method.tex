\chapter[Proposed Work]{Particle filter Based Map and Sensor Fusion technique for Indoor Tracking 
using an Android Smartphone \label{chap:proposed_method}}

As described in Chapter \ref{chap:literature_review} (Literature Review),
the step detection method for pedestrian navigation shows the most promise
for accurate tracking of subjects in indoor tracking scenarios. 
Hence, our proposed method will be building upon the same. However, specific
modifications are proposed and implemented to take into account the fact 
that our implementation system is an Android smartphone with a number of 
sensors and limited processing capabilities. 

The device for implementation is the Samsung Nexus S. The onboard accelerometer 
and magnetometer of the Nexus S are used as inertial navigation sensors.
Improvements are made to the algorithm keeping in mind the capabilities and
shortcomings of the mobile platform as well as the behaviour of sensors during
real world tests. The algorithm is then analyzed with
different parameters and approaches and is compared and contrasted with other
tracking approaches.

\section{Inputs}

Given the choice of hardware, we are constrained to use only specific 
data sources for our algorithm. The sections below detail the data 
sources used by the algorithm. Characterizations of these data sources
has already been done in Chapter \ref{chap:groundwork}

\subsection{Accelerometer}

\todo[inline]{Show the use of an Accelerometer to the system}

\section{What Doesn’t work}
Acceleration of the human body is too slow compared to gravity. Simple orientation changes produce high changes in acceleration along axes which are very difficult to filter out.
Numerical integration generates errors too quickly. With a high sampling rate, errors balloon.
Using the gyroscope for step detection (similar to foot mounted devices [insert reference]) – doesn’t work.


\subsection{Orientation}

\todo[inline]{Show the use of a Magnetometer to the system}

\subsection{Camera info\label{sec:QRCodes}}
QRCodes are a kind of 2 dimensional bar code
that can be read off from a camera. They can contain a variety of content - 
from URLs to Contact information. They can also contain plain text. The 
plain text information mode of QRCodes is leveraged to provide encoding 
of a human-readable text representation of map points. We can leverage 
the on-board camera of smartphones to provide information about the map 
of the floor, the current location and any additional information that is 
required by the tracking algorithm. Such factors might include the scaling 
factor of the map, Wifi AP information, navigation information etc.

\subsubsection{QRCode information format}

\todo[inline]{Add information about the message format for embedding data inside the QRCode}


\subsection{Map Information}
\todo[inline]{How is map info an input to the algorithm. Expand.}


\subsection{Wifi Information}

\todo[inline]{The case for using Wifi information to correct tracking drift}


\section{Dead reckoning with step counts}

The acceleration of the human body as part of the act of walking is small and 
very impulsive in nature. The motion of the torso of the body is governed 
mostly by inertia. Unfortunately, the impulsive acceleration values are small 
and lie buried in sensor noise. They are virtually indetectible using the MEMS 
accelerometer supplied with the android device - Nexus S.

To overcome this difficulty, the step count method of \todo{Provide reference}[ped. Nav.] is adopted. 
Under the assumption that step size varies very little over the course of 
motion of the subject, an inertial navigation system may be created. An
empirical formula is used to correlate acceleration values from 
the 

\subsection{Step counting method}

The step counting method is described in [ped. Nav.]. However, we take a
slightly modified approach. 

\todo[inline]{Write a short recap on the step counting approach}

\subsection{Noise clamping\label{sec:NoiseClamping}}

It is well known that while measuring real world information, sensors of all 
kinds pick up noise from the environment that affects the readings of the 
sensed variables. The MEMS accelerometer present on a typical smartphone is 
no different. Figure \ref{fig:accel_static} graphs the sensor noise of 
the accelerometer sensor for our device under test.

The noise level for the accelerometer was determined to be always less
than $0.6 m/s^2$. The acceleration peaks corresponding to actual steps usually
lie around $2.0 m/s^2$. To allow for a reasonable noise margin and provide
sufficient cushion for additional noise introduced due to the dynamic nature of
walking, we choose a threshold of $1.3 m/s^2$ which is the mean value of the two
peak values. If the absolute value of the Z-axis acceleration sample is less
than this threshold, then the sample will be clamped to zero. For a smartphone
with a similar accelerometer sensor but with different noise characteristics,
the values of the threshold can be varied accordingly. 

Figure \ref{fig:accel_raw} graphs how the filtering process works by
supressing sensor noise close to zero while allowing sensor values to 
pass unaltered when they correspond to a step being taken.

\subsection{Step detection procedure}

\subsubsection{Zero Crossings}

To detect actual steps taken by the subject holding the device, [ped. Nav.] 
suggests using zero crossings. However, in the sensor data collected, a number
of spurious peaks and valleys exist (primarily due to sensor noise). However, 
even after sensor noise is clamped as per Section \ref{sec:NoiseClamping}, 
spurious peaks and valleys that arise due to variable motion of the subject 
are not completely eliminated.

\subsubsection{Peak and Valley hunting}

Peak and Valley hunting procedure is an alternate method for step detection.
In this method a 2-state machine is constructed according to Table
\ref{tbl:peak_valley_state_table}. This method outputs a detected step at 
the first positive peak in the accelerometer sensor data that corresponds to 
the beginning of the next step. This is contrast to the zero-crossing method 
that outputs the detection of a step at every negative to positive zero 
crossing. 


An internal state machine is used. The state machine has 2 states and a comparison
is made between $A_{max}$ or $A_{min}$ and the sample value that forms the peak/trough
whenever state transitions occur.

\todo{Improve this area}
\begin{table}[h]\centering
    \caption{State table of the step detection state machine\label{tbl:peak_valley_state_table}}
    \begin{tabular}{cccc} \hline
    State & Accelerometer Value     & New State &  Action\\     \hline
    $q_0$ & Positive Peak Detected  & $q_1$     & Update $A_{max}$ if peak value  \\ 
          &                         &           & is positive and of \\
          &                         &           & a larger magnitude than \\
          &                         &           & current $A_{max}$ \\
          & Other values            & $q_0$     & Ignore \\         \hline
    $q_1$ & Negative Trough Detected & $q_0$    & Update $A_{min}$ if trough \\
          &                         &           & is negative and of larger \\ 
          &                         &           & magnitude than $A_{min}$ \\
          & Other values            & $q_1$     & Ignore \\ \hline
    \end{tabular}
\end{table}
\subsection{Step Size Estimation}

Reference [A. Engineers] provides this empirical relationship between acceleration
values and step size.

\begin{equation}\label{eq:step_size}
 Step-size = C \sqrt[4]{A_{max} - A_{min}}
\end{equation}

The constant $C$ is a scaling factor that is used as a constant of proportionality
to scale the step-values to real world distances and $A_{max}$ and $A_{min}$
represent maximum and minimum acceleration values corresponding to the 
peaks and troughs associated with a step.


\section{Determining the Training Constant}

An experimental method was used to determine the training constant for each
user. It involved the user to perform a short walk between 2 QRCoded locations
present in a straight line along a corridoor.
Since, the QRCodes represent fixed locations in the real world, the actual 
distance between them can be found using simple Euclidean geometry. From 
simple step counting, we are able to figure out the number of steps taken.
Also, by means of the step size formula mentioned in \eqref{eq:step_size},
we can estimate the distance travelled based purely on the accelerometer
values. We can then find the training constant by simply plugging in 
the values into the following equation:

\begin{equation}
C=\frac{\sqrt{(x_{2}-x_{1})^{2}+(y_{2}-y_{1})^{2}}}{\sum_{n=1}^{stepcount}\sqrt[4]{(A_{max_{i}}-A_{min_{i}})}}
\end{equation}

Here,\\
\begin{tabular}{l l}
$C$                         & is the training constant   \\
$(x_1, y_1), (x_2, y_2)$    & are the anchor point locations provided by the QRCodes \\
$A_{max_{i}}, A_{min_{i}}$  & represent the maximum and minimum \\
                            & acceleration values corresponding to the $i-th$ step.\\
\end{tabular}


\section{Dynamical equations for system}

Effectively, we can state that the basic system is being modelled by the
following set of equations:

\begin{equation}\label{eq:dr_eq}
\begin{bmatrix}x_{i+1}\\
y_{i+1}
\end{bmatrix} = \begin{bmatrix}x_{i}\\
y_{i}
\end{bmatrix}  + d{}_{i} \begin{bmatrix}-cos(\theta_{i})\\
sin(\theta_{i})
\end{bmatrix} 
\end{equation}

For the case where the coordinate system for the map is $x$ positive towards right and $y$  positive downwards with TrueNorth  of the map pointing upwards.

This dynamical representation is overtly simplistic because it doesn't take into account real world issues as seen in the groundwork section. The most important issue is the issue of sensor drift. The magnetometer is a rather inaccurate sensor and is rated to an accuracy of 5 degrees in static circumstances. There is also a recommendation to re-calibrate before use. This is required because this sensor suffers from a lot of sensor noise and drift. 

Recalibration of the magnetometer involves moving it around in a pattern of 8. Effectively, that randomizes internal magnetic elements enough for magnetic saturation effects to be neutralized. Unfortunately, for a continuous use scenario like ours, recalibration of this sensor is not an option. Hence, we modify our dynamical equation to take this error into account.

\begin{equation}
\begin{bmatrix}x_{i+1}\\
y_{i+1}
\end{bmatrix} = \begin{bmatrix}x_{i}\\
y_{i}
\end{bmatrix}  + d{}_{i} \begin{bmatrix}-cos(\theta_{i}+\vartheta)\\
sin(\theta_{i}+\vartheta)
\end{bmatrix} 
\end{equation}

In this dynamical equation, we have added an additional parameter $\vartheta$ which is a random variable that represents random white noise in the reading from the true value of the magnetometer.

Besides the sensor noise that creeps into the values of the magnetometer, there are 2 other issues that need to be taken care of in our dynamical modelling of the dead reckoning system.

\subsection{Accounting for orientation bias and noise}

The first issue is an issue of bias in the angle readings from the magnetometer. This bias can creep in due to 2 different reasons - the first being specific, environmental magnetic fields which distort the actual detection of $TrueNorth$  in the system and the second being a bias that creeps in due to the way the user holds the smartphone in the palm of his hand and the offset thus produced. To take into account such offsets, we modify the dynamical equations as follows:

\begin{equation}
\begin{bmatrix}x_{i+1}\\
y_{i+1}
\end{bmatrix} = \begin{bmatrix}x_{i}\\
y_{i}
\end{bmatrix}  + d{}_{i} \begin{bmatrix}-cos(\theta_{i}+\theta_{b}+\vartheta)\\
sin(\theta_{i}+\theta_{b}+\vartheta)
\end{bmatrix} 
\end{equation}

In this modified version of the dynamical equations, we have added a slowly
varying term $\theta_{b}$ that represents an explicit bias in the readings from
the magnetometer.

\subsection{Accounting for varying step sizes}

The second issue at hand is step size variation and missing steps. To map
accelerometer readings to step sizes, we have used the empirical equation
provided by [ref]. However, this empirical equation doesn't take into account
changes in step sizes due to changes in footwear or floor material. To account
for this bias in step size detection, we introduce an additional parameter
$d_{b}$ in the dynamical system. The new equations for the dynamical system are:

\begin{equation}
\begin{bmatrix}x_{i+1}\\
y_{i+1}
\end{bmatrix} = \begin{bmatrix}x_{i}\\
y_{i}
\end{bmatrix}  + (d{}_{i}+d_{b}) \begin{bmatrix}-cos(\theta_{i}+\theta_{b}+\vartheta)\\
sin(\theta_{i}+\theta_{b}+\vartheta)
\end{bmatrix} 
\end{equation}

In this representation, $d_{b}$ is a slowly varying bias variable on the step size.

\subsection{Limiting particle degeneracy}

\todo[inline]{Describe additional randomization added before next loop}

\section{Particle Filter Implementation of the Dynamical Equations}

There are a number of ways in which you can transform these dynamical 
equations to practice. An HMM based approach might be taken if a discrete
output space is preferred. A particle filter approach is taken when the 
output state space is continuous. Since we intend to keep the output space
continuous, we choose the Particle filter implementation of the dynamical
equations over the HMM implementation.

\subsection{Algorithm for updates}

\todo[inline]{Expand this section}

\section{Fusing Map Information}

\todo[inline]{Explain how map information can act as an error correcting 
    tool for the system.}

\section{Inherent limitations of the Particle filter approach}

\begin{enumerate}
\item Computational complexity
\item Possibility of running out of valid successor states (effectively getting lost)
\item Degeneracy of states
\end{enumerate}

\section{Heuristics to combat these limitations}

The following heuristics are used to combat some of the limitations 
described above.

\section{Future enhancements}
\subsection{Fusing Barometer Information}
\todo[inline]{Propose sensor fusion for barometric sensors as and when they become available}

