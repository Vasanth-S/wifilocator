\chapter{Summary and Future Work\label{chap:summary}}

\section{Particle filter + Wifi data}

\subsection{Case 1: Use of Wifi data for accounting for drift errors}
\subsection{Case 2: Use of Oriented Wifi data for accounting for drift errors}
\subsection{Case 3: Use of clamped wifi data for accounting for drift errors}
\subsection{Case 4: Use of area Restricted Wifi data for accounting for drift errors}
\subsection{Case 7: Simple post processing of output.??}


\section{Limitations}

User moving backwards. (Uses more computation, destroys the averaging procedure)


\section{Future enhancements}

There is a lot of scope for improving a number of aspects of the 
algorithms proposed. For example, we need to look at improving $MapSelect$,
integrating information from barometric sensors as and when they are made
available on handheld devices. We also need to figure out how to modify
the information to be put into first fix QRCodes to link up maps of 
the same building.

On the implementation front, performance of the solution, though 
adequate, can be significantly improved.
Improvements can be achieved by leveraging thread parallelism and a 
producer consumer model to allow for higher system lag in the case of 
a recovery event. The UI can be made event driven from the current 
polling model to increase responsiveness.


Use and evaluation of SVM as a suitable architecture for the Wifi data.
Better integration of information from the camera.
Integration of information using a Barometric sensor would be useful for detecting elevation changes. A system for the same is proposed but the experimental results were not obtained. 


\section{Signal Strength Map}

TODO if time permits
Wifi signal strength map developed using crowdsourced data for use in systems such as RedPin.


\section{Integration of barometric information}

\section{Exploration of crowdsourcing Wifi data}

\section{Analysis of Wifi signal strength distribution}


