%% bare_jrnl_compsoc.tex
%% V1.3
%% 2007/01/11
%% by Michael Shell
%% See:
%% http://www.michaelshell.org/
%% for current contact information.
%%
%% This is a skeleton file demonstrating the use of IEEEtran.cls
%% (requires IEEEtran.cls version 1.7 or later) with an IEEE Computer
%% Society journal paper.
%%
%% Support sites:
%% http://www.michaelshell.org/tex/ieeetran/
%% http://www.ctan.org/tex-archive/macros/latex/contrib/IEEEtran/
%% and
%% http://www.ieee.org/

%%*************************************************************************
%% Legal Notice:
%% This code is offered as-is without any warranty either expressed or
%% implied; without even the implied warranty of MERCHANTABILITY or
%% FITNESS FOR A PARTICULAR PURPOSE! 
%% User assumes all risk.
%% In no event shall IEEE or any contributor to this code be liable for
%% any damages or losses, including, but not limited to, incidental,
%% consequential, or any other damages, resulting from the use or misuse
%% of any information contained here.
%%
%% All comments are the opinions of their respective authors and are not
%% necessarily endorsed by the IEEE.
%%
%% This work is distributed under the LaTeX Project Public License (LPPL)
%% ( http://www.latex-project.org/ ) version 1.3, and may be freely used,
%% distributed and modified. A copy of the LPPL, version 1.3, is included
%% in the base LaTeX documentation of all distributions of LaTeX released
%% 2003/12/01 or later.
%% Retain all contribution notices and credits.
%% ** Modified files should be clearly indicated as such, including  **
%% ** renaming them and changing author support contact information. **
%%
%% File list of work: IEEEtran.cls, IEEEtran_HOWTO.pdf, bare_adv.tex,
%%                    bare_conf.tex, bare_jrnl.tex, bare_jrnl_compsoc.tex
%%*************************************************************************

% *** Authors should verify (and, if needed, correct) their LaTeX system  ***
% *** with the testflow diagnostic prior to trusting their LaTeX platform ***
% *** with production work. IEEE's font choices can trigger bugs that do  ***
% *** not appear when using other class files.                            ***
% The testflow support page is at:
% http://www.michaelshell.org/tex/testflow/



% Note that the a4paper option is mainly intended so that authors in
% countries using A4 can easily print to A4 and see how their papers will
% look in print - the typesetting of the document will not typically be
% affected with changes in paper size (but the bottom and side margins will).
% Use the testflow package mentioned above to verify correct handling of
% both paper sizes by the user's LaTeX system.
%
% Also note that the "draftcls" or "draftclsnofoot", not "draft", option
% should be used if it is desired that the figures are to be displayed in
% draft mode.
%
% The Computer Society usually requires 12pt for submissions.
%
\documentclass[10pt,journal,letterpaper,compsoc]{IEEEtran}
%
% If IEEEtran.cls has not been installed into the LaTeX system files,
% manually specify the path to it like:
% \documentclass[12pt,journal,compsoc]{../sty/IEEEtran}


% Some very useful LaTeX packages include:
% (uncomment the ones you want to load)


% *** MISC UTILITY PACKAGES ***
%
%\usepackage{ifpdf}
% Heiko Oberdiek's ifpdf.sty is very useful if you need conditional
% compilation based on whether the output is pdf or dvi.
% usage:
% \ifpdf
%   % pdf code
% \else
%   % dvi code
% \fi
% The latest version of ifpdf.sty can be obtained from:
% http://www.ctan.org/tex-archive/macros/latex/contrib/oberdiek/
% Also, note that IEEEtran.cls V1.7 and later provides a builtin
% \ifCLASSINFOpdf conditional that works the same way.
% When switching from latex to pdflatex and vice-versa, the compiler may
% have to be run twice to clear warning/error messages.






% *** CITATION PACKAGES ***
%
\ifCLASSOPTIONcompsoc
  % IEEE Computer Society needs nocompress option
  % requires cite.sty v4.0 or later (November 2003)
  \usepackage[nocompress]{cite}
\else
  % normal IEEE
  \usepackage{cite}
\fi
% cite.sty was written by Donald Arseneau
% V1.6 and later of IEEEtran pre-defines the format of the cite.sty package
% \cite{} output to follow that of IEEE. Loading the cite package will
% result in citation numbers being automatically sorted and properly
% "compressed/ranged". e.g., [1], [9], [2], [7], [5], [6] without using
% cite.sty will become [1], [2], [5]--[7], [9] using cite.sty. cite.sty's
% \cite will automatically add leading space, if needed. Use cite.sty's
% noadjust option (cite.sty V3.8 and later) if you want to turn this off.
% cite.sty is already installed on most LaTeX systems. Be sure and use
% version 4.0 (2003-05-27) and later if using hyperref.sty. cite.sty does
% not currently provide for hyperlinked citations.
% The latest version can be obtained at:
% http://www.ctan.org/tex-archive/macros/latex/contrib/cite/
% The documentation is contained in the cite.sty file itself.
%
% Note that some packages require special options to format as the Computer
% Society requires. In particular, Computer Society  papers do not use
% compressed citation ranges as is done in typical IEEE papers
% (e.g., [1]-[4]). Instead, they list every citation separately in order
% (e.g., [1], [2], [3], [4]). To get the latter we need to load the cite
% package with the nocompress option which is supported by cite.sty v4.0
% and later. Note also the use of a CLASSOPTION conditional provided by
% IEEEtran.cls V1.7 and later.





% *** GRAPHICS RELATED PACKAGES ***
%
\ifCLASSINFOpdf
  \usepackage[pdftex]{graphicx}
  % declare the path(s) where your graphic files are
  % \graphicspath{{../pdf/}{../jpeg/}}
  % and their extensions so you won't have to specify these with
  % every instance of \includegraphics
  \DeclareGraphicsExtensions{.pdf,.jpeg,.png}
\else
  % or other class option (dvipsone, dvipdf, if not using dvips). graphicx
  % will default to the driver specified in the system graphics.cfg if no
  % driver is specified.
  \usepackage[dvips]{graphicx}
  % declare the path(s) where your graphic files are
  % \graphicspath{{../eps/}}
  % and their extensions so you won't have to specify these with
  % every instance of \includegraphics
  \DeclareGraphicsExtensions{.eps}
\fi
% graphicx was written by David Carlisle and Sebastian Rahtz. It is
% required if you want graphics, photos, etc. graphicx.sty is already
% installed on most LaTeX systems. The latest version and documentation can
% be obtained at: 
% http://www.ctan.org/tex-archive/macros/latex/required/graphics/
% Another good source of documentation is "Using Imported Graphics in
% LaTeX2e" by Keith Reckdahl which can be found as epslatex.ps or
% epslatex.pdf at: http://www.ctan.org/tex-archive/info/
%
% latex, and pdflatex in dvi mode, support graphics in encapsulated
% postscript (.eps) format. pdflatex in pdf mode supports graphics
% in .pdf, .jpeg, .png and .mps (metapost) formats. Users should ensure
% that all non-photo figures use a vector format (.eps, .pdf, .mps) and
% not a bitmapped formats (.jpeg, .png). IEEE frowns on bitmapped formats
% which can result in "jaggedy"/blurry rendering of lines and letters as
% well as large increases in file sizes.
%
% You can find documentation about the pdfTeX application at:
% http://www.tug.org/applications/pdftex





% *** MATH PACKAGES ***
%
\usepackage[cmex10]{amsmath}
% A popular package from the American Mathematical Society that provides
% many useful and powerful commands for dealing with mathematics. If using
% it, be sure to load this package with the cmex10 option to ensure that
% only type 1 fonts will utilized at all point sizes. Without this option,
% it is possible that some math symbols, particularly those within
% footnotes, will be rendered in bitmap form which will result in a
% document that can not be IEEE Xplore compliant!
%
% Also, note that the amsmath package sets \interdisplaylinepenalty to 10000
% thus preventing page breaks from occurring within multiline equations. Use:
%\interdisplaylinepenalty=2500
% after loading amsmath to restore such page breaks as IEEEtran.cls normally
% does. amsmath.sty is already installed on most LaTeX systems. The latest
% version and documentation can be obtained at:
% http://www.ctan.org/tex-archive/macros/latex/required/amslatex/math/





% *** SPECIALIZED LIST PACKAGES ***
%
\usepackage{algorithmic}
% algorithmic.sty was written by Peter Williams and Rogerio Brito.
% This package provides an algorithmic environment fo describing algorithms.
% You can use the algorithmic environment in-text or within a figure
% environment to provide for a floating algorithm. Do NOT use the algorithm
% floating environment provided by algorithm.sty (by the same authors) or
% algorithm2e.sty (by Christophe Fiorio) as IEEE does not use dedicated
% algorithm float types and packages that provide these will not provide
% correct IEEE style captions. The latest version and documentation of
% algorithmic.sty can be obtained at:
% http://www.ctan.org/tex-archive/macros/latex/contrib/algorithms/
% There is also a support site at:
% http://algorithms.berlios.de/index.html
% Also of interest may be the (relatively newer and more customizable)
% algorithmicx.sty package by Szasz Janos:
% http://www.ctan.org/tex-archive/macros/latex/contrib/algorithmicx/




% *** ALIGNMENT PACKAGES ***
%
%\usepackage{array}
% Frank Mittelbach's and David Carlisle's array.sty patches and improves
% the standard LaTeX2e array and tabular environments to provide better
% appearance and additional user controls. As the default LaTeX2e table
% generation code is lacking to the point of almost being broken with
% respect to the quality of the end results, all users are strongly
% advised to use an enhanced (at the very least that provided by array.sty)
% set of table tools. array.sty is already installed on most systems. The
% latest version and documentation can be obtained at:
% http://www.ctan.org/tex-archive/macros/latex/required/tools/


%\usepackage{mdwmath}
%\usepackage{mdwtab}
% Also highly recommended is Mark Wooding's extremely powerful MDW tools,
% especially mdwmath.sty and mdwtab.sty which are used to format equations
% and tables, respectively. The MDWtools set is already installed on most
% LaTeX systems. The lastest version and documentation is available at:
% http://www.ctan.org/tex-archive/macros/latex/contrib/mdwtools/


% IEEEtran contains the IEEEeqnarray family of commands that can be used to
% generate multiline equations as well as matrices, tables, etc., of high
% quality.


%\usepackage{eqparbox}
% Also of notable interest is Scott Pakin's eqparbox package for creating
% (automatically sized) equal width boxes - aka "natural width parboxes".
% Available at:
% http://www.ctan.org/tex-archive/macros/latex/contrib/eqparbox/





% *** SUBFIGURE PACKAGES ***
%\ifCLASSOPTIONcompsoc
%\usepackage[tight,normalsize,sf,SF]{subfigure}
%\else
%\usepackage[tight,footnotesize]{subfigure}
%\fi
% subfigure.sty was written by Steven Douglas Cochran. This package makes it
% easy to put subfigures in your figures. e.g., "Figure 1a and 1b". For IEEE
% work, it is a good idea to load it with the tight package option to reduce
% the amount of white space around the subfigures. Computer Society papers
% use a larger font and \sffamily font for their captions, hence the
% additional options needed under compsoc mode. subfigure.sty is already
% installed on most LaTeX systems. The latest version and documentation can
% be obtained at:
% http://www.ctan.org/tex-archive/obsolete/macros/latex/contrib/subfigure/
% subfigure.sty has been superceeded by subfig.sty.


%\ifCLASSOPTIONcompsoc
%  \usepackage[caption=false]{caption}
%  \usepackage[font=normalsize,labelfont=sf,textfont=sf]{subfig}
%\else
%  \usepackage[caption=false]{caption}
%  \usepackage[font=footnotesize]{subfig}
%\fi
% subfig.sty, also written by Steven Douglas Cochran, is the modern
% replacement for subfigure.sty. However, subfig.sty requires and
% automatically loads Axel Sommerfeldt's caption.sty which will override
% IEEEtran.cls handling of captions and this will result in nonIEEE style
% figure/table captions. To prevent this problem, be sure and preload
% caption.sty with its "caption=false" package option. This is will preserve
% IEEEtran.cls handing of captions. Version 1.3 (2005/06/28) and later 
% (recommended due to many improvements over 1.2) of subfig.sty supports
% the caption=false option directly:
%\ifCLASSOPTIONcompsoc
%  \usepackage[caption=false,font=normalsize,labelfont=sf,textfont=sf]{subfig}
%\else
%  \usepackage[caption=false,font=footnotesize]{subfig}
%\fi
%
% The latest version and documentation can be obtained at:
% http://www.ctan.org/tex-archive/macros/latex/contrib/subfig/
% The latest version and documentation of caption.sty can be obtained at:
% http://www.ctan.org/tex-archive/macros/latex/contrib/caption/




% *** FLOAT PACKAGES ***
%
%\usepackage{fixltx2e}
% fixltx2e, the successor to the earlier fix2col.sty, was written by
% Frank Mittelbach and David Carlisle. This package corrects a few problems
% in the LaTeX2e kernel, the most notable of which is that in current
% LaTeX2e releases, the ordering of single and double column floats is not
% guaranteed to be preserved. Thus, an unpatched LaTeX2e can allow a
% single column figure to be placed prior to an earlier double column
% figure. The latest version and documentation can be found at:
% http://www.ctan.org/tex-archive/macros/latex/base/



%\usepackage{stfloats}
% stfloats.sty was written by Sigitas Tolusis. This package gives LaTeX2e
% the ability to do double column floats at the bottom of the page as well
% as the top. (e.g., "\begin{figure*}[!b]" is not normally possible in
% LaTeX2e). It also provides a command:
%\fnbelowfloat
% to enable the placement of footnotes below bottom floats (the standard
% LaTeX2e kernel puts them above bottom floats). This is an invasive package
% which rewrites many portions of the LaTeX2e float routines. It may not work
% with other packages that modify the LaTeX2e float routines. The latest
% version and documentation can be obtained at:
% http://www.ctan.org/tex-archive/macros/latex/contrib/sttools/
% Documentation is contained in the stfloats.sty comments as well as in the
% presfull.pdf file. Do not use the stfloats baselinefloat ability as IEEE
% does not allow \baselineskip to stretch. Authors submitting work to the
% IEEE should note that IEEE rarely uses double column equations and
% that authors should try to avoid such use. Do not be tempted to use the
% cuted.sty or midfloat.sty packages (also by Sigitas Tolusis) as IEEE does
% not format its papers in such ways.




%\ifCLASSOPTIONcaptionsoff
%  \usepackage[nomarkers]{endfloat}
% \let\MYoriglatexcaption\caption
% \renewcommand{\caption}[2][\relax]{\MYoriglatexcaption[#2]{#2}}
%\fi
% endfloat.sty was written by James Darrell McCauley and Jeff Goldberg.
% This package may be useful when used in conjunction with IEEEtran.cls'
% captionsoff option. Some IEEE journals/societies require that submissions
% have lists of figures/tables at the end of the paper and that
% figures/tables without any captions are placed on a page by themselves at
% the end of the document. If needed, the draftcls IEEEtran class option or
% \CLASSINPUTbaselinestretch interface can be used to increase the line
% spacing as well. Be sure and use the nomarkers option of endfloat to
% prevent endfloat from "marking" where the figures would have been placed
% in the text. The two hack lines of code above are a slight modification of
% that suggested by in the endfloat docs (section 8.3.1) to ensure that
% the full captions always appear in the list of figures/tables - even if
% the user used the short optional argument of \caption[]{}.
% IEEE papers do not typically make use of \caption[]'s optional argument,
% so this should not be an issue. A similar trick can be used to disable
% captions of packages such as subfig.sty that lack options to turn off
% the subcaptions:
% For subfig.sty:
% \let\MYorigsubfloat\subfloat
% \renewcommand{\subfloat}[2][\relax]{\MYorigsubfloat[]{#2}}
% For subfigure.sty:
% \let\MYorigsubfigure\subfigure
% \renewcommand{\subfigure}[2][\relax]{\MYorigsubfigure[]{#2}}
% However, the above trick will not work if both optional arguments of
% the \subfloat/subfig command are used. Furthermore, there needs to be a
% description of each subfigure *somewhere* and endfloat does not add
% subfigure captions to its list of figures. Thus, the best approach is to
% avoid the use of subfigure captions (many IEEE journals avoid them anyway)
% and instead reference/explain all the subfigures within the main caption.
% The latest version of endfloat.sty and its documentation can obtained at:
% http://www.ctan.org/tex-archive/macros/latex/contrib/endfloat/
%
% The IEEEtran \ifCLASSOPTIONcaptionsoff conditional can also be used
% later in the document, say, to conditionally put the References on a 
% page by themselves.




% *** PDF, URL AND HYPERLINK PACKAGES ***
%
%\usepackage{url}
% url.sty was written by Donald Arseneau. It provides better support for
% handling and breaking URLs. url.sty is already installed on most LaTeX
% systems. The latest version can be obtained at:
% http://www.ctan.org/tex-archive/macros/latex/contrib/misc/
% Read the url.sty source comments for usage information. Basically,
% \url{my_url_here}.





% *** Do not adjust lengths that control margins, column widths, etc. ***
% *** Do not use packages that alter fonts (such as pslatex).         ***
% There should be no need to do such things with IEEEtran.cls V1.6 and later.
% (Unless specifically asked to do so by the journal or conference you plan
% to submit to, of course. )


% correct bad hyphenation here
\hyphenation{op-tical net-works semi-conduc-tor}


\begin{document}
%
% paper title
% can use linebreaks \\ within to get better formatting as desired
\title{A Particle Filter based Indoor Tracking Solution\\on an Android Smartphone}
%
%
% author names and IEEE memberships
% note positions of commas and nonbreaking spaces ( ~ ) LaTeX will not break
% a structure at a ~ so this keeps an author's name from being broken across
% two lines.
% use \thanks{} to gain access to the first footnote area
% a separate \thanks must be used for each paragraph as LaTeX2e's \thanks
% was not built to handle multiple paragraphs
%
%
%\IEEEcompsocitemizethanks is a special \thanks that produces the bulleted
% lists the Computer Society journals use for "first footnote" author
% affiliations. Use \IEEEcompsocthanksitem which works much like \item
% for each affiliation group. When not in compsoc mode,
% \IEEEcompsocitemizethanks becomes like \thanks and
% \IEEEcompsocthanksitem becomes a line break with idention. This
% facilitates dual compilation, although admittedly the differences in the
% desired content of \author between the different types of papers makes a
% one-size-fits-all approach a daunting prospect. For instance, compsoc 
% journal papers have the author affiliations above the "Manuscript
% received ..."  text while in non-compsoc journals this is reversed. Sigh.

\author{Divye~Kapoor,~\IEEEmembership{Member,~IEEE,}
        Dr.~Manoj~Misra,~\IEEEmembership{Member,~IEEE}% <-this % stops a space
\IEEEcompsocitemizethanks{\IEEEcompsocthanksitem D. Kapoor and Dr. M. Misra are with the Department
of Electronics and Computer Engineering, Indian Institute of Technology, Roorkee, UKD, 247667.\protect\\
% note need leading \protect in front of \\ to get a newline within \thanks as
% \\ is fragile and will error, could use \hfil\break instead.
E-mail: divyekapoor@gmail.com}% <-this % stops a space
\thanks{Manuscript received \textbf{April 19, 2005}; revised \textbf{January 11, 2007}.}}

% note the % following the last \IEEEmembership and also \thanks - 
% these prevent an unwanted space from occurring between the last author name
% and the end of the author line. i.e., if you had this:
% 
% \author{....lastname \thanks{...} \thanks{...} }
%                     ^------------^------------^----Do not want these spaces!
%
% a space would be appended to the last name and could cause every name on that
% line to be shifted left slightly. This is one of those "LaTeX things". For
% instance, "\textbf{A} \textbf{B}" will typeset as "A B" not "AB". To get
% "AB" then you have to do: "\textbf{A}\textbf{B}"
% \thanks is no different in this regard, so shield the last } of each \thanks
% that ends a line with a % and do not let a space in before the next \thanks.
% Spaces after \IEEEmembership other than the last one are OK (and needed) as
% you are supposed to have spaces between the names. For what it is worth,
% this is a minor point as most people would not even notice if the said evil
% space somehow managed to creep in.



% The paper headers
\markboth{Journal of \LaTeX\ Class Files,~Vol.~6, No.~1, January~2007}%
{Shell \MakeLowercase{\textit{et al.}}: Bare Demo of IEEEtran.cls for Computer Society Journals}
% The only time the second header will appear is for the odd numbered pages
% after the title page when using the twoside option.
% 
% *** Note that you probably will NOT want to include the author's ***
% *** name in the headers of peer review papers.                   ***
% You can use \ifCLASSOPTIONpeerreview for conditional compilation here if
% you desire.



% The publisher's ID mark at the bottom of the page is less important with
% Computer Society journal papers as those publications place the marks
% outside of the main text columns and, therefore, unlike regular IEEE
% journals, the available text space is not reduced by their presence.
% If you want to put a publisher's ID mark on the page you can do it like
% this:
%\IEEEpubid{0000--0000/00\$00.00~\copyright~2007 IEEE}
% or like this to get the Computer Society new two part style.
%\IEEEpubid{\makebox[\columnwidth]{\hfill 0000--0000/00/\$00.00~\copyright~2007 IEEE}%
%\hspace{\columnsep}\makebox[\columnwidth]{Published by the IEEE Computer Society\hfill}}
% Remember, if you use this you must call \IEEEpubidadjcol in the second
% column for its text to clear the IEEEpubid mark (Computer Society jorunal
% papers don't need this extra clearance.)



% use for special paper notices
%\IEEEspecialpapernotice{(Invited Paper)}



% for Computer Society papers, we must declare the abstract and index terms
% PRIOR to the title within the \IEEEcompsoctitleabstractindextext IEEEtran
% command as these need to go into the title area created by \maketitle.
\IEEEcompsoctitleabstractindextext{%
\begin{abstract}
%\boldmath
This paper discusses the implementation of a modified particle filter 
based indoor tracking system which leverages map and site information 
to yield sub metre accuracy of tracking on a standard Android 
smartphone. Specific challenges addressed by this paper include 
the determination and correction for orientation bias of the user due to the way
the smartphone is held by her and the magnetic bias present in smartphone 
readings due to stray environmental magnetic fields as well as due to 
sensor memory effects. This paper also addresses the concerns of 
prior authors regarding modifications required for sensor fusing 
Wifi information with particle filters on account of the high computational
complexity of particle filters.
\end{abstract}
% IEEEtran.cls defaults to using nonbold math in the Abstract.
% This preserves the distinction between vectors and scalars. However,
% if the journal you are submitting to favors bold math in the abstract,
% then you can use LaTeX's standard command \boldmath at the very start
% of the abstract to achieve this. Many IEEE journals frown on math
% in the abstract anyway. In particular, the Computer Society does
% not want either math or citations to appear in the abstract.

% Note that keywords are not normally used for peerreview papers.
\begin{IEEEkeywords}
Android, Smartphone, Indoor Tracking, Sensor Bias, User Bias
\end{IEEEkeywords}}


% make the title area
\maketitle


% To allow for easy dual compilation without having to reenter the
% abstract/keywords data, the \IEEEcompsoctitleabstractindextext text will
% not be used in maketitle, but will appear (i.e., to be "transported")
% here as \IEEEdisplaynotcompsoctitleabstractindextext when compsoc mode
% is not selected <OR> if conference mode is selected - because compsoc
% conference papers position the abstract like regular (non-compsoc)
% papers do!
\IEEEdisplaynotcompsoctitleabstractindextext
% \IEEEdisplaynotcompsoctitleabstractindextext has no effect when using
% compsoc under a non-conference mode.


% For peer review papers, you can put extra information on the cover
% page as needed:
% \ifCLASSOPTIONpeerreview
% \begin{center} \bfseries EDICS Category: 3-BBND \end{center}
% \fi
%
% For peerreview papers, this IEEEtran command inserts a page break and
% creates the second title. It will be ignored for other modes.
\IEEEpeerreviewmaketitle



%\section{Introduction}
% Computer Society journal papers do something a tad strange with the very
% first section heading (almost always called "Introduction"). They place it
% ABOVE the main text! IEEEtran.cls currently does not do this for you.
% However, You can achieve this effect by making LaTeX jump through some
% hoops via something like:
%
\ifCLASSOPTIONcompsoc
  \noindent\raisebox{2\baselineskip}[0pt][0pt]%
  {\parbox{\columnwidth}{\section{Introduction}\label{sec:introduction}%
  \global\everypar=\everypar}}%
  \vspace{-1\baselineskip}\vspace{-\parskip}\par
\else
  \section{Introduction}\label{sec:introduction}\par
\fi
%
% Admittedly, this is a hack and may well be fragile, but seems to do the
% trick for me. Note the need to keep any \label that may be used right
% after \section in the above as the hack puts \section within a raised box.



% The very first letter is a 2 line initial drop letter followed
% by the rest of the first word in caps (small caps for compsoc).
% 
% form to use if the first word consists of a single letter:
% \IEEEPARstart{A}{demo} file is ....
% 
% form to use if you need the single drop letter followed by
% normal text (unknown if ever used by IEEE):
% \IEEEPARstart{A}{}demo file is ....
% 
% Some journals put the first two words in caps:
% \IEEEPARstart{T}{his demo} file is ....
% 
% Here we have the typical use of a "T" for an initial drop letter
% and "HIS" in caps to complete the first word.
\IEEEPARstart{S}{martphones} have started becoming extremely powerful and
versatile due to recent developments in technology. This has opened up a new
class of applications that were previoiusly presumed to be impossible due to
either computational constraints or due to physical device limitations.
Specifically, the introduction of fast processors (with clock speeds of the 
order of 1GHz and more) allow many more computations to be performed within 
a limited time frame. This is very important for near-realtime applications 
like indoor tracking. 


% You must have at least 2 lines in the paragraph with the drop letter
% (should never be an issue)
% For right alignment
%I wish you the best of success.

%\hfill mds
 
%\hfill January 11, 2007

%\subsection{Subsection Heading Here}
%Subsection text here.

%-------------------- Start Content ------------------------------------
\section{Relevant Prior Work}
As described in Chapter \ref{chap:literature_review} (Literature Review),
the pedestrian navigation system described in \cite{Wang} 
shows the most promise for accurate tracking of subjects in indoor tracking scenarios. 
The results of \cite{Evennou} which implements a similar system integrating an INS 
with Wifi data are very encouraging. However, we are cautioned 
against the computational complexity of a particle filter based solution and its attendant
implications while implementing the same on a mobile device by Evennou and Marx\cite{Evennou}. As we shall later
see, these cautionary words are well founded.

Our proposed method thus builds upon the work of Evennou and Marx\cite{Evennou}
and that of Wang et al\cite{Wang}. However many modifications specific to
implementation on a resource constrained device like an Android smartphone are
made. The proposed device for implementation is the Samsung I9020 (also called
the Samsung Nexus S which is co-branded with Google).

%Specifically, we discard the use of Wifi information for the central tracking
%algorithm and instead use it only in the error recovery phase. The justifications
%for these modifications are made in the accompanying text.
%modifications are proposed and implemented to take into account the fact 
%that our implementation system is an Android smartphone with a number of 
%sensors and limited processing capabilities. 

%The device that is the test-bed for implementation is the Samsung Nexus S. 
%The onboard accelerometer, magnetometer and gyroscope of the Nexus S are used 
%as inertial navigation sensors. The algorithm is then analyzed with different 
%parameters and paths and is compared and contrasted with two other tracking 
%approaches implemented on the same device - one a simple uncorrected dead
%reckoning implementation and the other a pure Wifi based based tracking system.
%These results are summarized in Chapter \ref{chap:results}.

\section{System Inputs\label{sec:system_inputs}}

Given the choice of hardware, we are constrained to use only specific 
data sources for our algorithm. 
The sensors available on a typical Android Smartphone are:
\begin{enumerate}
\item A 3-axis accelerometer for measuring acceleration of the device.
\item A 3-axis magnetometer for measuring the surrounding magnetic field.
\item A gyroscope for providing angular velocity information.
\item A camera for providing visual information.
\item A Wifi Network Card with the ability to provide measurements of the received signal strengths. (RSSI)
\end{enumerate}

Besides the sensors mentioned above, additional sources of information 
that we can make available to the system are:

\begin{enumerate}
\item Map information detailing the test environment.
\item A site survey of the test environment which provides Wifi location fingerprints augmented by 
    orientation information of the reciever when the fingerprint was taken.
\end{enumerate}

Characterizations of these data sources as well
as other relevant parameters will be done as ground-work in 
Chapter \ref{chap:groundwork}.

\section{Device Limitations\label{sec:device_limitations}}

Being a resource constrained device, a number of limitations are imposed on 
software for these systems. Some of the ones most affecting the implementation
are:

\begin{itemize}
\item Applications are restricted to a maximum memory utilization of 16 MB on 
    older devices and newer devices get limits based on the size of their 
    RAM. The Nexus S has a maximum limit of 32 MB, other devices usually have
    lower limits of 24 MB or 16 MB.
\item The CPU usage of the application has to be controlled. Applications 
    that use the CPU intensively for more than a few seconds are detected as 
    misbehaving applications and the user is prompted to kill them.
\item Sensor events are dropped if the event delivery thread is busy processing
    the last delivered event. This puts significant limits on the 
    amount of inter sample processing that may be performed.
\end{itemize}


\section{Background on Particle Filters}

Particle filters belong to a class of Sequential Monte Carlo algorithms 
that depend essentially approximating a probability distribution function based
on a Monte Carlo simulation of the system using observation samples. Particle 
filters have been known to be very useful for tracking in systems where joint
probability distribution functions are either not completely known or hard 
to sample. 

In this case, our problem of indoor tracking is essentially the 
determination of a latent variable $x_t$ which represents the position of the 
user after we receive a series of observations ($o_0 \dots o_t$) from the 
sensors on the Android device. Effectively, we wish to approximate the the 
probablity distribution $p(x_t|o_t,o_{t-1}\dots,o_0)$ at any time instant $t$
so that we may determine the position of the device using the expectation 
function $E[x]$ on the probability distribution $p(x_t|o_t,o_{t-1}\dots,o_0)$
and we wish to use a particle filter to help us do so.

Particle filters are a huge topic in and of themselves. Thus, in the interests
of brevity, the reader is referred to the excellent expositions of particle 
filters for tracking applications in \cite{Ristic} and to \cite{Arulampalam}
for particle filters in the context of non-linear, non-gaussian Bayesian
tracking.


\section{Proposed Method}

Keeping the availability of the inputs described above as well as the device 
limitations described in Section \ref{sec:device_limitations} in mind, a less 
computationally intensive method is proposed in 
Algorithm \ref{algo:proposed_method}.

%%%%
% TODO
%%%%
%\begin{algorithm}
%\dontprintsemicolon
%\SetKwInOut{Input}{Input}
%\Input{Sensor Events from device sensors}
%\KwResult{An estimate of the location of the device at any time through the variable $L$}
%\Begin{
%    $L \longleftarrow FirstFix()$ \;
%    $P \longleftarrow$ $N$ particles around $L$ corrupted by gaussian noise\;
%    \ForEach{$stepEvent$ from $sensors$}{
%        $P' \longleftarrow ApplyDynamicalEquations(P)$\;
%        $P' \longleftarrow MapSelect(P', P)$\;
%        \eIf{$P'$ is $\emptyset$}{
%            $R \longleftarrow PerfomRecovery(P)$\;
%            \If{$R$ is not $\emptyset$}{
%                $P \longleftarrow R$
%            }
%        }{
%            $P \longleftarrow P'$\;
%        }
%        Resample set $P$ uniformly to a fixed sized set and
%        corrupt the new particles thus generated with gaussian noise\;
%        $L \longleftarrow Mean(P)$\;
%    }
%}
%\caption{High Level View of the System\label{algo:proposed_method}}
%\end{algorithm}

Algorithm \ref{algo:proposed_method} is very general. Also, though it is 
stated in a sequential manner, the it needs to 
be implemented in an event-driven fashion to allow for online tracking. 
The \textbf{ForEach} loop in the algorithm is essentially an event loop 
that depends on step detection.

Procedures to determine \emph{FirstFix}, detect steps from sensor data, 
\emph{MapSelect}, \emph{ApplyDynamicalEquations}, and \emph{PerfomRecovery} are explained
below.

%\begin{enumerate}
%\item A first-fix position estimate will be taken by means of manual input or 
%    via QRCodes through the smartphone camera.
%\item A step counting approach in accordance with a modified peak and 
%    valley approach and a step stride estimation in accordance with
%    \cite{ADXL202} will be used.
%\item A low complexity particle filter is used to implement the tracking 
%    algorithm. 
%\item Map information is used to provide means for correcting drift in the 
%    simple tracking algorithm.
%\item Additional "hidden" variables are used in the dynamical equations of the 
%    system to compensate for a constant sensor drift and small variations in 
%    the step size of the user. 
%\end{enumerate}

Comparison of the proposed method will be done with 2 other algorithms, also
implemented on the smartphone: 

\begin{enumerate}
\item A simple step based dead reckoning solution that directly uses sensor data.
\item A simple step based dead reckoning solution with a Wifi based nearest neighbour algorithm incorporated as a 
    drift correction method.    
\end{enumerate}

These algorithms are being implemented as control algorithms for comparisons
in order to provide suitable comparative information since maps and test 
environments differ substantially and make comparative analysis difficult.

\section{First Fix\label{sec:first_fix}}

Dead reckoning refers to a method of position determination where the current
location is estimated based on a known starting location and offsets from the
known location measured via instruments.

For any dead reckoning algorithm, it is important to provide a good starting
point. Low error in the starting location contributes to better performance.
Our proposed method is based on the dead reckoning concept and thus requires 
a (reasonably) accurate starting point (also called a first fix).

In order to provide the starting point to the algorithm two approaches are 
proposed:

\begin{enumerate}
\item Directly choose the starting location on a map using the capacitive 
    touchscreen
\item Select the starting location based on QRCode recognition using the onboard
    camera. The QRCodes were printed and pasted at specific locations 
    in the test environment and the user is required to simply take a 
    picture of it using the onboard camera.
\end{enumerate}

\subsection{QR Codes\label{sec:QRCodes}}
QRCodes are a kind of 2 dimensional machine readable code (an example of which
is shown in Figure \ref{fig:sample_qrcode}). The information encoded in this
format can be read off by processing an image acquired from a camera. 

QRCodes can contain a variety of content - 
from URLs to Contact information. They can also contain plain text. 

For this application, we use the plain text information mode of QRCodes 
to provide encoding of a human-readable text representation of map points. 
This human and machine readable representation is achieved with the lightweight
JavaScript Object Notation (JSON) representation. 
Details of the format chosen for this application are
mentioned below:

\begin{figure}
    \centering
    \includegraphics[width=3in]{figures/sample_qrcode}
    \caption{A Sample QRCode\label{fig:sample_qrcode}}
\end{figure}


\subsubsection{QR Code information format}

A typical format used for storing all the relevant information in a QRCode is
shown in Figure \ref{fig:QRCode_info_format}. As is proper for any data 
serialization format, it stores a Version number and a Type field to 
distinguish itself from other data serialization formats. It also includes
critical information about the point itself - the X and Y locations on 
the relevant map as scaled on a 0-1 scale. A full description of the fields
is given in Table \ref{tbl:QRCode_fields_table}.

The fields are descriptive 
and quite human readable. However, because they are in JSON, a basic 
check of well-formedness can be made by programs. The data format retains 
extensibility as it is able to accommodate additional fields which will
be ignored by applications that are not built to expect the presence of those
fields.

\begin{figure}
    \centering
\begin{verbatim}
{
   "Version": 1,
   "Type": "MapPoint",
   "MapURL": "http://www.iitr.ernet.in/path/to/sample/map.png",
   "Scale": 0.010716161,
   "X":0.70625,
   "Y":0.16479166666666667 
   
}
\end{verbatim}
    \caption{A sample data point encoded as text information in a QRCode\label{fig:QRCode_info_format}}
\end{figure}

\begin{table}
\centering
\begin{tabular}{p{1in} p{4in}}
\hline
\hline
Field Name      &       Field Description \\
\hline
Version         & Constant value 1. May be incremented if additional fields are added to the format. \\
Type            & Represents the type of QRCode waypoint sample that was just acquired. Type "MapPoint" indicates that the QRCode represents a point on a Map. Alternative and  additional information can be provided by choosing other Types. For example,  a Type of "WifiAP" could be used to provide location information about  Wifi Access Points in the surroundings. \\
MapURL          & This field always refers to a publicly available map associated with the environment where  the QRCode was pasted. A png map type has been indicated, but any other resource type that  a client is able to handle should be accepted. \\
Scale           & Scale is an optional parameter of a MapPoint. It represents the scaling factor between distances on a map with distances in the real world. It can be omitted if it is expected that the  client will have some other way of figuring out the scale of the map based on information encoded within the map itself. \\
X and Y         & These are values encoded in the real range [0-1] which represent the location of the QRCode on a map referred by it. These values are used for providing a first fix to the  reckoning algorithms.  \\
\hline
\end{tabular}
\caption{Explanation of the fields used in the QRCode Information Format\label{tbl:QRCode_fields_table}}
\end{table}

\section{Noise filtering\label{sec:NoiseClamping}}

It is well known that while measuring real world information, most sensors
 pick up noise from the environment that affects the readings of the 
sensed variables. The MEMS accelerometer present on a typical smartphone is 
no different. 

In general, we assume that the accelerometer sensor noise is below a small 
threshold $T$ when the device is static on a firm surface such as a table. 
A higher threshold $Q$ is required when the device is being kept at the palm of 
a hand because of small noisy variations introduced by the palm itself.

The filtering process uses a simple clamping mechanism. The filter 
rejects a reading of the accelerometer based on the following constraints:

\begin{equation}
NoiseFilter(a_i) =  \begin{cases} 0 & \text{if $|a_i| \le Q$,} \\
                                a_i & \text{otherwise}
                    \end{cases}
\end{equation}

The values of T and Q are determined as part of the groundwork in Chapter 
\ref{chap:groundwork}.

The filtering process is very important for the step counting algorithms
to detect accurate peaks.

\section{Distance Estimation}

The acceleration of the human body as part of the act of walking is small and 
very impulsive in nature. Unfortunately, the linear acceleration values
produced by it are so small that they are virtually indetectible from the 
sensor noise of the MEMS accelerometer of the Android device. However, the 
act of putting a foot on the ground generates an impulse along the Z 
axis of the accelerometer which is clearly distinguishable in the sensor 
readings. Thus, we use the step count method described in
\cite{Wang} and \cite{Ladetto} to detect motion of the user holding the device.

By relating the acceleration impulses to the size of a step, 
an inertial navigation system may be created. An empirical formula 
from \cite{ADXL202} is used to estimate step sizes from the readings sensed by
the accelerometer. Details of all the methods used to achieve the same 
follow:

\subsection{Step detection procedure\label{sec:step_detection}}

\subsubsection{Zero Crossings}

This step counting method is described in \cite{Wang} is based on a similar 
method devised for outdoor pedestrian navigation by \cite{Ladetto}. It is a
simple formulation that counts the number of zero crossings of a 
filtered version of the raw accelerometer signal which represent double the 
number of steps taken. However, this method, when implemented, generates 
step events at zero crossings which correspond to a body in motion.
Since each step has to be associated with an angle of motion, it might be 
advantageous to ensure that step events are associated with points where 
the foot of the user is on the ground, leading to better stability of angle 
readings. Thus, the Peak and Valley hunting method is also proposed. 

\subsubsection{Peak and Valley hunting\label{sec:peak_and_valley}}

The Peak and Valley hunting procedure is an alternate method being
proposed for step detection.
This method will detect the same number of crossings as the zero crossing 
method but will raise the step events whenever the foot of the user 
strikes the ground for the beginning of the next step. 

In this method a 2-state machine is constructed according to Table
\ref{tbl:peak_valley_state_table}. This method outputs a detected step at 
the first positive peak in the accelerometer sensor data that corresponds to 
the beginning of the next step. Also, since it only processes data at 
points where the first derivative of the accelerometer signal is zero and 
the accelerometer signal itself is non-zero, it can reduce computational 
effort for cases where zero crossing would perform considerable testing
on account of being at the value 0, waiting for a transition. Additionally,
the values of $A_{max}$ and $A_{min}$ that will be required later are 
updated only during these intervals, thus saving further computational 
resources compared the Zero Crossing method. The disadvantage of this method
though is the additional lag introduced between a step being taken and its 
event being delivered to the application.

\begin{table}[h]\centering
    \begin{tabular}{c p{1in} c p{2.7in}} \hline
    State & Accel. Value     & New State &  Action\\     \hline
    $q_0$ & Positive Peak ($P$) Detected  & $q_1$     & Update $A_{max}$ if peak $P > A_{max}$. Output a step event if $flag = 1$. Reset $flag$ to 0.  \\ 
          & Other values            & $q_0$     & Ignore \\         \hline
    $q_1$ & Negative Trough ($T$) Detected & $q_0$    & Update $A_{min}$ if $T < A_{min}$. Set $flag$ to 1. \\
          & Other values            & $q_1$     & Ignore \\ \hline
    \end{tabular}
    \caption{State table of the step detection state machine\label{tbl:peak_valley_state_table}}
\end{table}

\subsection{Step Size Estimation}

Engineers from Analog Device have published an empirical relationship between 
acceleration values and step size in \cite{ADXL202}:

\begin{equation}\label{eq:step_size}
 Step-size = C \sqrt[4]{A_{max} - A_{min}}
\end{equation}

The constant $C$ is a scaling factor that is used as a constant of proportionality
to scale the step-values to real world distances and $A_{max}$ and $A_{min}$
represent maximum and minimum acceleration values corresponding to the 
peaks and troughs associated with a step. 

This equation is used to obtain an estimate of the step size when a user 
takes a step.

\subsection{Determining the Training Constant}

The training constant for each user was determined experimentally. 
Users were asked to perform a short walk between 2 QRCoded locations
present in a straight line along a corridoor.
Since, the QRCodes represent fixed locations in the real world, the actual 
distance between them can be found using simple Euclidean geometry. From 
simple step counting, we are able to figure out the number of steps taken.
Also, by means of the step size formula mentioned in \eqref{eq:step_size},
we can estimate the distance travelled based purely on the accelerometer
values. We can then find the training constant by simply plugging in 
the values into the following equation:

\begin{equation}
C=\frac{\sqrt{(x_{2}-x_{1})^{2}+(y_{2}-y_{1})^{2}}}{\sum_{i=1}^{stepcount}\sqrt[4]{(A_{max_{i}}-A_{min_{i}})}}
\end{equation}

Here,\\
\begin{tabular}{l l}
$C$                         & is the training constant   \\
$(x_1, y_1), (x_2, y_2)$    & are the anchor point locations provided by the QRCodes \\
$A_{max_{i}}, A_{min_{i}}$  & represent the maximum and minimum \\
                            & acceleration values corresponding to the $i^{th}$ step.\\
$stepcount$                 & The number of steps detected by the algorithm in section \ref{sec:step_detection} \\
\end{tabular}


\section{Dynamical Equations for the System}

With the preliminaries now out of the way, we can develop the dynamical model
of our proposed solution. The basic step update equation for the system 
can be written as:

\begin{equation}\label{eq:dr_eq}
\begin{bmatrix}x_{i+1}\\
y_{i+1}
\end{bmatrix} = \begin{bmatrix}x_{i}\\
y_{i}
\end{bmatrix}  + d{}_{i} \begin{bmatrix}-cos(\theta_{i})\\
sin(\theta_{i})
\end{bmatrix} 
\end{equation}

Here,\\
\begin{tabular}{p{1in} p{4in}}
$x_i, y_i$          &   represent location of the device after the $i^{th}$ step\\
$x_0, y_0$          &   are the first fix values obtained via the methods of Section \ref{sec:first_fix}\\
$d_i$               &   the predicted step distance as per the step estimate equation \eqref{eq:step_size}\\
$\theta_i$          &   the angle associated with the detected step based on a magnetometer reading\\
\end{tabular}

These equations are written assuming a coordinate system for the map where the 
origin is at the top left corner of the map, the $x$ axis being
positive towards right and $y$ axis positive downwards with $TrueNorth$ of the map
pointing upwards. Effectively, equation \eqref{eq:dr_eq} represents a raw dead reckoning
solution. Unfortunately, these equations do nothing to counteract the biggest
disadvantage of dead reckoning: unbounded error growth. The dynamical
representation is also overtly simplistic because it fails to take into account
other issues such as sensor drift and sensor bias. The magnetometer is the
biggest culprit in this regard as it is highly sensitive to stray magnetic
effects in the environment.

\subsection{Modelling Orientation Sensor Noise}

\begin{equation}
\begin{bmatrix}x_{i+1}\\
y_{i+1}
\end{bmatrix} = \begin{bmatrix}x_{i}\\
y_{i}
\end{bmatrix}  + d{}_{i} \begin{bmatrix}-cos(\theta_{i}+\vartheta)\\
sin(\theta_{i}+\vartheta)
\end{bmatrix} 
\end{equation}

We modify the dynamical equation to add an additional parameter $\vartheta$
which is a random variable that represents random variations akin to white noise
in the reading from the true value of the magnetometer. $\theta_i$ now
represents the true angle associated with the step motion.

\subsection{Accounting for Orientation Sensor Bias\label{sec:angle_bias}}

Angular readings from the magnetometer often turn out to be biased from
$TrueNorth$. This bias can creep in due to 2 different reasons - the first being
specific, environmental magnetic fields which distort the actual detection of
$TrueNorth$ in the system and the second being a bias that creeps in due to the
way the user holds the smartphone in the palm of his hand and the offset thus
produced. To take into account such offsets, we modify the dynamical equations
as follows:

\begin{equation}
\begin{bmatrix}x_{i+1}\\
y_{i+1}\\
\theta_{b_{i+1}}
\end{bmatrix} = \begin{bmatrix}x_{i}\\
y_{i}\\
\theta_{b_i}
\end{bmatrix}  + \begin{bmatrix} d{}_{i} & d_i & 1 \end{bmatrix} \begin{bmatrix}-cos(\theta_{i}+\theta_{b_i}+\vartheta)\\
sin(\theta_{i}+\theta_{b_i}+\vartheta)\\
\theta_t\\
\end{bmatrix} 
\end{equation}

\begin{equation}
\theta_t \sim \mathcal{N}(0,\sigma_{angle})
\end{equation}

In this modified version of the dynamical equations, we have added a slowly
varying term $\theta_{b_i}$ that represents an explicit bias in the readings
from the magnetometer at the time of the $i^{th}$. This bias value is updated at
each step by altering it with a small gaussian noise of zero mean an a variance
determined by the characteristics of the magnetometer. The choice of
$\sigma_{angle}$ was made on the basis of the groundwork done. Corrections
for the sensor bias will be made when $MapSelect$ is applied to the particles by
the overall algorithm.

\subsection{Accounting for Varying Step Sizes and Missing Steps\label{sec:step_bias}}

The second issue at hand is step size variation and missing steps. Steps may
be missed by the step detection procedures or phantom steps may be detected
due to sensor noise or other events. Changes in step sizes due to changes in 
footwear or floor material as well as the leads and lags produced due to a 
slightly incorrect calibration constant also contribute to deviations of 
apparent distance travelled versus the apparent distance travelled. To account
for this bias in step size detection, we introduce an additional parameter
$d_{b}$ in the dynamical system that accounts for the step bias. 
Thus, the new equations for the dynamical system are:

\begin{equation}
\begin{bmatrix}x_{i+1}\\
y_{i+1}\\
\theta_{b_{i+1}}\\
d_{b_{i+1}}\\
\end{bmatrix} = \begin{bmatrix}x_{i}\\
y_{i}\\
\theta_{b_i}\\
d_{b_i}
\end{bmatrix}  + \begin{bmatrix}(d{}_{i}+d_{b_i}) & (d{}_{i}+d_{b_i}) & 1 & 1\end{bmatrix} \begin{bmatrix}-cos(\theta_{i}+\theta_{b_i}+\vartheta)\\
sin(\theta_{i}+\theta_{b_i}+\vartheta)\\
\theta_t\\
d_{t}
\end{bmatrix} 
\end{equation}

\begin{equation}
d_{t} \sim \mathcal{N}(0, \sigma_{step})
\end{equation}

In this representation, $d_{b_i}$ is a slowly varying bias variable on the step
size and the gaussian variance used during each update step is $d_t$. The 
determination of the variance of $d_t$ again depends on sensor to sensor and 
is evaluated empirically for our system to be close to $step\_size/25$.

Note that the equations mentioned in Sections \ref{sec:angle_bias} and
\ref{sec:step_bias} will work only if the map information is dense enough to
eventually eliminate groups of particles with incorrect orientations and 
step drifts. Otherwise, the only effect of maintaining the bias variables will 
be to slightly increase the variance of the variables $x_i$ and $y_i$.
An explanation of how these bias variables actually help determine overall
angle and step bias is provided in Section \ref{sec:overall_bias}

\subsection{Ensuring particle diversity}

The presence of the variables $\theta_{t}$ and $d_t$ represents an 
intentional introduction of randomness into the system to ensure 
particle diversity. For example, if two particles start out with the same
state vectors, say 
$ [ x_i y_i \theta_{b_i} d_{b_i} ] $ and 
$ [ x_j y_j \theta_{b_j} d_{b_j} ] $, they will in the immediate 
next time step tend to diverge due to possibly different samples of the 
random variables $\theta_{t}$ and $d_t$. So, 
when evolved over a large number of steps, similar samples will take 
slightly different paths through the state space, thus ensuring particle 
diversity of the particle filter. Note however, that the choice of 
magnitudes of the variance of $\theta_t$ and $d_t$ is critical. If the 
values chosen are too small, states that were possible physically are 
not achieved by the particle filter and if the values chosen are too large,
computational effort will be wasted for unachievable states. For our case 
of a highly resource constrained machine, this is very undesirable for we
wish to maximize the effectiveness of our particles. In this regard, it is 
also very advantageous if we have a highly detailed map of the area for 
unachievable states are quickly pruned and computational effort is 
redirected towards states with more likelihood. The integration of 
map information is described in the subsequent section.

\section{Integrating Map Information}

Map information can be integrated into the system in a number 
of ways. The simplest way to use map information as a selection 
function which is similar to \cite{Wang}:

\begin{equation}\label{eq:select}
MapSelect(p_{i+1}, p_i) = \begin{cases}1 - P_{wall} & \text{if $p_i + \alpha (p_{i+1} - p_i)$ for $\alpha \in [0,1]$} \\
                                                    & \text{does not cross a wall}\\
                                    P_{wall} & \text{otherwise}
                          \end{cases}
\end{equation}

Here,\\
\begin{tabular}{p{1.5in} p{3.5in}}
$p_{i+1}$ and $p_i$ &   are tuples corresponding to map locations ($x_{i+1}$, $y_{i+1}$) and ($x_i$, $y_i$) respectively. \\
$P_{wall}$ &   represents the probability of selection of the particle if it crosses a wall. \\
$\alpha$ & is a parametric variable used to represent a line between the two points $p_{i+1}$ and $p_i$ \\
\end{tabular}

Wang et all\cite{Wang} suggest that a $P_{wall}$ of 0 be used as such a motion
is impossible. Though simplistic, this approach works well by swiftly removing
unreachable particles. It is very light computationally too. However, given our
very limited budget for particles, it might not be prudent to remove a particle
simply because it was crossing a wall very close to a door. Thus, more advanced
map integration techniques may well be envisaged with $P_{wall}$ varying based
on the distance of the intersection point from the closest door in the wall. 

\section{Determination of overall orientation and step biases\label{sec:overall_bias}}

The integration of map information with the dynamical equations of the system 
via a selection function $MapSelect$ allows us to actually determine 
the extent of orientation and step bias. The idea behind it is simple. 
However, before explaining it, let us define the following two terms:
\begin{equation}
\displaystyle
\theta_b = \frac{1}{N}\sum\limits_{i} \theta_{b_i}
\end{equation}

\begin{equation}
\displaystyle
d_b = \frac{1}{N}\sum\limits_{i} d_{b_i}
\end{equation}

Here,\\
\begin{tabular}{p{1.5in} p{3.5in}}
N & is the total number of live particles in the system \\
$d_{b_i}$ and $\theta_{b_i}$ & retain their meanings from the previous sections \\
$\theta_b$ & represents overall system orientation bias at the $i^{th}$ stage \\
$d_b$ & represents overall system step bias at the $i^{th}$ stage \\
\end{tabular}

When no walls exist in the vicinity of the particles, map information is sparse
and few particles are eliminated because of it. Thus, we have insufficient 
feedback from our system to determine the value of orientation bias ($\theta_b$)
as all positions are equally valid. However, in a confined space like a 
doorway or a corridoor, there exist only a few valid range of angles over 
a sequence of steps. Thus
errant particles will be swiftly rejected by $MapSelect$ and we will 
get a highly accurate estimate of the degree of orientation bias present 
in our system.

The justification for step bias ($d_b$) lies similarly in the fact that while
moving along corridoors, we will have accumulated
a certain degree of step drift. This drift ensures that we are slightly 
uncertain of our exact position along the corridoor after walking through it for a long time
as we have not received any feedback from $MapSelect$ along our direction of motion. 
Thus,
some particles representing our location will be further along the corridoor than others and we have no way 
to choose between them other than by treating them at par. However, 
whenever we make a sharp turn on the map, the presence of walls and other 
obstacles immediately before and after the junctions on the map will cause 
a number of particles to be rejected by $MapSelect$. The values of $d_{b_i}$
remaining represent the net bias in the step size of the system. Thus,
their mean $d_b$ is a good representation of the step bias of the system.

\section{Recovering from particle insufficiency}

Particle insufficiency is a weakness of particle filters. It arises if 
no valid successor states $p_{i+1}$ can be found from the current set of 
particles $p_i$ when the observation $o_i$ is taken into account and applied 
to the dynamical equations of the system. (That is, all successor states 
$p_{i+1}$, have $MapSelect(p_{i+1}, p_i) = 0$). This possibility is 
further accentuated in resource constrained systems like ours which 
have a hard limit on the maximum number of particles that can be 
updated between two steps (since the algorithm is working in an online
fashion with a location estimate being available after every step).

The insufficiency of particles can be handled by a number of ways. 
The simplest one is: increase the number of particles being updated at each 
step and ensure enough particle diversity so that you always have 
a few particles that survive incorrect decisions of the particle filter. 

Widyawan \cite{Widyawan} suggests an alternative method: maintaining particle
history information so that an incorrect decision of the particle filter can be
rolled back a few steps in the past and an alternative decision can be made that
does not cause the particle filter to suffer particle insufficiency. However,
the method suggested is expensive in terms of memory usage and computational
cost.

The recovery algorithm ($PerformRecovery$) being proposed is stated in 
Algorithm \ref{algo:recovery}.

%%%%
%% TODO
%%%%
%\begin{algorithm}
%\dontprintsemicolon
%\linesnumbered
%\SetKw{Break}{break}
%\SetKwInOut{Input}{Input}
%\Input{A set of particles $P$ such that $ApplyDynamicalEquations(P)$ yields $\emptyset$, $\sigma_X$ and $\sigma_Y$ are the positioning errors of the Wifi based positioning system}
%\KwResult{A new set of particles $R$ which approximate the location of the device or $\emptyset$}
%\Begin{
%    \emph{Try to recover by finding particles in the surroundings by random sampling}\;
%    $retryCount \longleftarrow 0$\;
%    \While{$retryCount < MaxRetries$}{
%        \For{$j$ in $[1..N]$}{
%            $r_j \longleftarrow (\mathcal{N}(0, \sigma_X), \mathcal{N}(0, \sigma_Y))$\;
%            $selectParticle \longleftarrow false$\;
%            \ForEach{$p_i \in P$}{
%                \If{$MapSelect(r_j, p_i)$}{
%                    $selectParticle \longleftarrow true$\;
%                    \Break\;
%                }
%            }
%            \If{$selectParticle$ is $true$}{
%                $R \longleftarrow R \cup \{ r_j \}$\;
%            }
%        }
%        $R' \longleftarrow ApplyDynamicalEquations(R)$\;
%        \If{$R'$ is not $\emptyset$}{
%            \Return $R$\;
%        }
%        $retryCount \longleftarrow retryCount + 1$\;
%    }
%    \emph{We have failed to find suitable particles by random sampling, fall back on Wifi}\;
%    $wifiLocation \longleftarrow PerformWifiPositioning()$\;
%    \For{$j$ in $[1..N]$}{
%        \emph{Corrupt the wifi location with gaussian noise to approximate the error}\;
%        $r_j \longleftarrow wifiLocation + (\mathcal{N}(0, \sigma_X), \mathcal{N}(0, \sigma_Y))$\;
%        $R \longleftarrow R \cup \{ r_j \}$\;
%    }
%    $R' \longleftarrow ApplyDynamicalEquations(R)$\;
%    \eIf{$R'$ is not $\emptyset$}{
%        \Return $R$\;
%    }{
%        \emph{We've failed to find suitable particles, return failure and handle it by skipping the step information.}\;
%        \Return{$\emptyset$}\;
%    }
%}
%\caption{Recovery algorithm for particle insufficiency\label{algo:recovery}}
%\end{algorithm}

There are a few interesting things to note about the recovery algorithm.
The first being that it retries finding suitable particles. This is an 
optimization as we don't want to spend too much effort at recovering 
unless we really have to since the inner algorithm has an O($N \times |P|$) 
complexity. In case we fail to recover directly from random sampling 
around the location, we retry with Wifi Positioning data. If that fails 
too, then we return an empty set. The system handles this failure by skipping 
the step event (discarding it as possibly invalid sensor readings).

\section{Summary}

The above algorithms represent a lower complexity particle filter that 
is proposed to be suitable for implementation on resource constrained 
smartphones. The performance of the proposed methods is analyzed in 
Chapter \ref{chap:results}.


%--------------------   End Content ------------------------------------


% needed in second column of first page if using \IEEEpubid
%\IEEEpubidadjcol

\subsubsection{Subsubsection Heading Here}
Subsubsection text here.


% An example of a floating figure using the graphicx package.
% Note that \label must occur AFTER (or within) \caption.
% For figures, \caption should occur after the \includegraphics.
% Note that IEEEtran v1.7 and later has special internal code that
% is designed to preserve the operation of \label within \caption
% even when the captionsoff option is in effect. However, because
% of issues like this, it may be the safest practice to put all your
% \label just after \caption rather than within \caption{}.
%
% Reminder: the "draftcls" or "draftclsnofoot", not "draft", class
% option should be used if it is desired that the figures are to be
% displayed while in draft mode.
%
%\begin{figure}[!t]
%\centering
%\includegraphics[width=2.5in]{myfigure}
% where an .eps filename suffix will be assumed under latex, 
% and a .pdf suffix will be assumed for pdflatex; or what has been declared
% via \DeclareGraphicsExtensions.
%\caption{Simulation Results}
%\label{fig_sim}
%\end{figure}

% Note that IEEE typically puts floats only at the top, even when this
% results in a large percentage of a column being occupied by floats.
% However, the Computer Society has been known to put floats at the bottom.


% An example of a double column floating figure using two subfigures.
% (The subfig.sty package must be loaded for this to work.)
% The subfigure \label commands are set within each subfloat command, the
% \label for the overall figure must come after \caption.
% \hfil must be used as a separator to get equal spacing.
% The subfigure.sty package works much the same way, except \subfigure is
% used instead of \subfloat.
%
%\begin{figure*}[!t]
%\centerline{\subfloat[Case I]\includegraphics[width=2.5in]{subfigcase1}%
%\label{fig_first_case}}
%\hfil
%\subfloat[Case II]{\includegraphics[width=2.5in]{subfigcase2}%
%\label{fig_second_case}}}
%\caption{Simulation results}
%\label{fig_sim}
%\end{figure*}
%
% Note that often IEEE papers with subfigures do not employ subfigure
% captions (using the optional argument to \subfloat), but instead will
% reference/describe all of them (a), (b), etc., within the main caption.


% An example of a floating table. Note that, for IEEE style tables, the 
% \caption command should come BEFORE the table. Table text will default to
% \footnotesize as IEEE normally uses this smaller font for tables.
% The \label must come after \caption as always.
%
%\begin{table}[!t]
%% increase table row spacing, adjust to taste
%\renewcommand{\arraystretch}{1.3}
% if using array.sty, it might be a good idea to tweak the value of
% \extrarowheight as needed to properly center the text within the cells
%\caption{An Example of a Table}
%\label{table_example}
%\centering
%% Some packages, such as MDW tools, offer better commands for making tables
%% than the plain LaTeX2e tabular which is used here.
%\begin{tabular}{|c||c|}
%\hline
%One & Two\\
%\hline
%Three & Four\\
%\hline
%\end{tabular}
%\end{table}


% Note that IEEE does not put floats in the very first column - or typically
% anywhere on the first page for that matter. Also, in-text middle ("here")
% positioning is not used. Most IEEE journals use top floats exclusively.
% However, Computer Society journals sometimes do use bottom floats - bear
% this in mind when choosing appropriate optional arguments for the
% figure/table environments.
% Note that, LaTeX2e, unlike IEEE journals, places footnotes above bottom
% floats. This can be corrected via the \fnbelowfloat command of the
% stfloats package.



\section{Conclusion}
The conclusion goes here.





% if have a single appendix:
%\appendix[Proof of the Zonklar Equations]
% or
%\appendix  % for no appendix heading
% do not use \section anymore after \appendix, only \section*
% is possibly needed

% use appendices with more than one appendix
% then use \section to start each appendix
% you must declare a \section before using any
% \subsection or using \label (\appendices by itself
% starts a section numbered zero.)
%


\appendices
\section{Proof of the First Zonklar Equation}
Appendix one text goes here.

% you can choose not to have a title for an appendix
% if you want by leaving the argument blank
\section{}
Appendix two text goes here.


% use section* for acknowledgement
\ifCLASSOPTIONcompsoc
  % The Computer Society usually uses the plural form
  \section*{Acknowledgments}
\else
  % regular IEEE prefers the singular form
  \section*{Acknowledgment}
\fi


The authors would like to thank...


% Can use something like this to put references on a page
% by themselves when using endfloat and the captionsoff option.
\ifCLASSOPTIONcaptionsoff
  \newpage
\fi



% trigger a \newpage just before the given reference
% number - used to balance the columns on the last page
% adjust value as needed - may need to be readjusted if
% the document is modified later
%\IEEEtriggeratref{8}
% The "triggered" command can be changed if desired:
%\IEEEtriggercmd{\enlargethispage{-5in}}

% references section

% can use a bibliography generated by BibTeX as a .bbl file
% BibTeX documentation can be easily obtained at:
% http://www.ctan.org/tex-archive/biblio/bibtex/contrib/doc/
% The IEEEtran BibTeX style support page is at:
% http://www.michaelshell.org/tex/ieeetran/bibtex/
\bibliographystyle{IEEEtran}
% argument is your BibTeX string definitions and bibliography database(s)
\bibliography{IEEEabrv,mybib}
%
% <OR> manually copy in the resultant .bbl file
% set second argument of \begin to the number of references
% (used to reserve space for the reference number labels box)
%\begin{thebibliography}{1}
%
%\bibitem{IEEEhowto:kopka}
%H.~Kopka and P.~W. Daly, \emph{A Guide to \LaTeX}, 3rd~ed.\hskip 1em plus
%  0.5em minus 0.4em\relax Harlow, England: Addison-Wesley, 1999.
%
%\end{thebibliography}

% biography section
% 
% If you have an EPS/PDF photo (graphicx package needed) extra braces are
% needed around the contents of the optional argument to biography to prevent
% the LaTeX parser from getting confused when it sees the complicated
% \includegraphics command within an optional argument. (You could create
% your own custom macro containing the \includegraphics command to make things
% simpler here.)
%\begin{biography}[{\includegraphics[width=1in,height=1.25in,clip,keepaspectratio]{mshell}}]{Michael Shell}
% or if you just want to reserve a space for a photo:

\begin{IEEEbiography}{Divye Kapoor}
Divye Kapoor is a Masters graduate in Information Technology from the 
Indian Institute of Technology, Roorkee. 
\end{IEEEbiography}

% if you will not have a photo at all:
\begin{IEEEbiographynophoto}{John Doe}
Biography text here.
\end{IEEEbiographynophoto}

% insert where needed to balance the two columns on the last page with
% biographies
%\newpage

\begin{IEEEbiographynophoto}{Jane Doe}
Biography text here.
\end{IEEEbiographynophoto}

% You can push biographies down or up by placing
% a \vfill before or after them. The appropriate
% use of \vfill depends on what kind of text is
% on the last page and whether or not the columns
% are being equalized.

%\vfill

% Can be used to pull up biographies so that the bottom of the last one
% is flush with the other column.
%\enlargethispage{-5in}



% that's all folks
\end{document}


